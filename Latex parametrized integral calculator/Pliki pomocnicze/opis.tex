
% !TeX TXS-program:compile = txs:///pythontex
\documentclass[a4paper]{article}
\usepackage[utf8]{inputenc}
\usepackage{tikz}
\usepackage{tikz-cd}
\usepackage{pgfplots}
\usepackage{pythontex}
%biblioteka potrzebna do wyświetlenia ą,ę, itd.
\usepackage{polski}
%biblioteka z formatowaniem matematycznym
\usepackage{amsmath}
\usepackage{mathalfa}
\usepackage{amsfonts}

\section{całka oznaczona}
\textbf{Całka oznaczona} {\displaystyle \int \limits _{a}^{b}f(x)dx}
Intuicyjnie całka oznaczona to pole powierzchni między wykresem funkcji {\displaystyle f(x)} 
w pewnym przedziale {\displaystyle [a,b]} a osią odciętych, wzięte ze znakiem plus dla dodatnich wartości funkcji i minus dla ujemnych. 
Pojęcie całki oznaczonej, choć intuicyjnie proste, może być sformalizowane na wiele sposobów. 
Jeśli jakaś funkcja jest całkowalna według dwóch różnych definicji całki oznaczonej, wynik całkowania będzie taki sam.

\section{całka nieoznaczona}	
\textbf{Całka nieoznaczona} {\displaystyle \int f(x)dx} 
Przez całkę nieoznaczoną, albo funkcję pierwotną rozumie się pojęcie odwrotne do pochodnej funkcji. Całkę oznaczoną na przedziale {\displaystyle [a,b]} 
można też zdefiniować (tzw. całka Newtona-Leibniza) jako różnicę między wartościami całki nieoznaczonej w punktach {\displaystyle b} oraz {\displaystyle a.} 
Stąd obliczenie całki nieoznaczonej jest często pierwszym krokiem przy obliczaniu całek oznaczonych.
Uogólnieniem całki nieoznaczonej jest całka równania różniczkowego będąca rozwiązaniem równania różniczkowego: {\displaystyle F'(x)=f(x),} 
gdzie {\displaystyle F(x)} jest pierwotną, a {\displaystyle f(x)} oznacza całkowaną funkcję.
W drugiej połowie XX wieku wprowadzono nowe rodzaje całek nieoznaczonych, które umożliwiają obliczenia w obszarze analizy niearchimedesowej. 
Jedną z nich jest całka Volkenborna, określona przez granicę
{\displaystyle \int _{\mathbb {Z} _{p}}f(x)\,\mathrm {d} x=\lim _{n\to \infty }{\frac {1}{p^{n}}}\sum _{x=0}^{p^{n}-1}f(x).}

\section{całka Riemanna}
\textbf{Całka Riemanna}
Niech dana będzie funkcja ograniczona {\displaystyle f\colon [a,b]\to \mathbb {R} .} 
Sumą częściową Riemanna nazywa się liczbę {\displaystyle R_{f,P(q_{1},\dots ,q_{n})}=\sum _{i=1}^{n}f(q_{i})\cdot \Delta p_{i}.}
Funkjcę {\displaystyle f} nazywa się całkowalną w sensie Riemanna lub krótko R-całkowalną, jeśli dla dowolnego ciągu normalnego {\displaystyle (P^{k})} 
podziałów przedziału {\displaystyle [a,b],} istnieje (niezależna od wyboru punktów pośrednich) granica {\displaystyle R_{f}=\lim _{k\to \infty }R_{f,P^{k}\left(q_{1}^{k},\dots ,q_{n_{k}}^{k}\right)}}
nazywana wtedy całką Riemanna tej funkcji. 
Równoważnie: jeżeli istnieje taka liczba {\displaystyle R_{f},} że dla dowolnej liczby rzeczywistej {\displaystyle \varepsilon >0} 
istnieje taka liczba rzeczywista  {\displaystyle \delta >0,} że dla dowolnego podziału {\displaystyle P(q_{1},\dots ,q_{n})} 
o średnicy {\displaystyle \mathrm {diam} \;P(q_{1},\dots ,q_{n})<\delta ;} 
bądź też w języku rozdrobnień: że dla dowolnej liczby rzeczywistej {\displaystyle \varepsilon >0} 
istnieje taki podział {\displaystyle S(t_{1},\dots ,t_{m})} 
przedziału {\displaystyle [a,b],} że dla każdego podziału {\displaystyle P(q_{1},\dots ,q_{n})} 
rozdrabniającego {\displaystyle S(t_{1},\dots ,t_{m})} 
zachodzi{\displaystyle \left|R_{f,P(q_{1},\dots ,q_{n})}-R_{f}\right|<\varepsilon .}
Funkcję {\displaystyle f} nazywa się wtedy całkowalną w sensie Riemanna (R-całkowalną), a liczbę {\displaystyle R_{f}} jej całką Riemanna.

\section{całka Darboux}
\textbf{Całka Darboux, górna i dolna}
Niech dana będzie funkcja ograniczona {\displaystyle f\colon [a,b]\to \mathbb {R} .} 
Kresy dolny i górny funkcji {\displaystyle f} w danym podprzedziale {\displaystyle P_{i}} 
podziału {\displaystyle P} przedziału {\displaystyle [a,b]} oznaczane będą odpowiednio symbolami
{\displaystyle m_{f,P_{i}}=\inf _{x\in P_{i}}f(x)\quad {\text{ oraz }}\quad M_{f,P_{i}}=\sup _{x\in P_{i}}f(x);}
różnicę tych liczb {\displaystyle \omega _{f,P_{i}}=M_{f,P_{i}}-m_{f,P_{i}}}
nazywa się oscylacją funkcji {\displaystyle f} na przedziale {\displaystyle P_{i}.}
Odpowiednio sumą dolną i górną (Darboux) nazywa się liczby
{\displaystyle L_{f,P}=\sum _{i=1}^{n}m_{f,P_{i}}\cdot \Delta p_{i}\quad {\text{ oraz }}\quad U_{f,P}=\sum _{i=1}^{n}M_{f,P_{i}}\cdot \Delta p_{i}.}
Wielkości te umożliwiają zdefiniowanie całki dolnej i górnej Darboux funkcji {\displaystyle f} jako odpowiednio
{\displaystyle L_{f}=\sup {\big \{}L_{f,P}\colon P{\text{ jest podziałem }}[a,b]{\big \}}}
oraz {\displaystyle U_{f}=\inf {\big \{}U_{f,P}\colon P{\text{ jest podziałem }}[a,b]{\big \}}.}
O funkcji {\displaystyle f} mówi się, że jest całkowalna w sensie Darboux lub krótko D-całkowalną, jeżeli 
{\displaystyle L_{f}=U_{f};} wówczas tę wspólną wartość {\displaystyle D_{f}} 
całki dolnej i górnej Darboux nazywa się po prostu całką Darboux.
