% !TeX TXS-program:compile = txs:///pythontex
\documentclass[a4paper]{article}
\usepackage{geometry}  
\geometry{a4paper, total={170mm, 257mm}, left=20mm, top=20mm,}
\usepackage[utf8]{inputenc}
\usepackage{tikz}
\usepackage{tikz-cd}
\usepackage{pgfplots}
\usepackage{pythontex}
\usepackage{multirow}
%biblioteka potrzebna do wyświetlenia ą,ę, itd.
\usepackage{polski}
%biblioteka z formatowaniem matematycznym
\usepackage{amsmath}
\usepackage{mathalfa}
\usepackage{amsfonts}
\usepackage[T1]{fontenc}

\title{\textbf{Analiza matematyczna --- całkowanie dowolnej funkcji --- porównanie całkowania dolnego, górnego, numerycznego (przez trapezy) z całką oznaczoną (całka Riemanna).}}
\author{J. Ć., W. D., Maciej S.}
\date{05.02.2024}

\begin{document}
\maketitle
\vspace{3cm}
\newtheorem{definition}{Definicja}

\begin{pycode}
from wykresy import *
#-------------------------------------------
#	Parametryzacja

#y = f(x)
funkcja = "8/(19*x**2) + x**2 - 4 - sympy.sin(x)"
	
#na ile elementów dzielimy odcinek
n = 15

#wybierz przedział całkowania (a, b)
a = 1
b = 10
	
#-------------------------------------------

def f(x):
	return sympy.sympify(str(funkcja))

x = sympy.symbols("x")
funkcja = eval(funkcja)

RIEMANN = riemann_integral(f, 'x', a, b, n, 'riemann.pgf')[0]
DARBOUX = darboux_sums(f, 'x', a, b, n, 'darboux.pgf')
TRAPEZOID = trapezoid_rule(f, 'x', a, b, n, 'trapez.pgf')[0]

def trunc(input):
	return '{:0.2f}'.format(input)

\end{pycode}

\section{Wprowadzenie merytoryczne}
\subsection{Całka nieoznaczona}
\begin{definition}
	Jeśli $F(x)$ jest funkcją pierwotną $f(x)$ na przedziale $I$, to wyrażenie $F(x) + C$ nazywamy całką nieoznaczoną funkcji $f(x)$ na $I$ i oznaczamy: $$\int f(x)dx = F(x) + C,$$ gdzie $C$ to dowolna stała.
\end{definition}
Przez całkę nieoznaczoną albo funkcję pierwotną rozumie się pojęcie odwrotne do pochodnej funkcji. Stąd obliczenie całki nieoznaczonej jest często pierwszym krokiem przy obliczaniu całek oznaczonych. W drugiej połowie XX wieku wprowadzono nowe rodzaje całek nieoznaczonych, które umożliwiają obliczenia w obszarze analizy niearchimedesowej. Jedną z nich jest całka Volkenborna, określona przez granicę:
{$$\displaystyle \int _{\mathbb {Z} _{p}}f(x)\,\mathrm {d} x=\lim _{n\to \infty }{ {1}{p^{n}}}\sum _{x=0}^{p^{n}-1}f(x)$$}.

\subsection{Całka oznaczona (w sensie Riemanna)}

Intuicyjnie całka oznaczona to pole powierzchni między wykresem funkcji {$\displaystyle f(x)$} 
w pewnym przedziale {$\displaystyle [a, b]$} a osią odciętych, wzięte ze znakiem plus dla dodatnich wartości funkcji i minus dla ujemnych. Pojęcie całki oznaczonej, choć intuicyjnie proste, może być sformalizowane na wiele sposobów. Jeśli jakaś funkcja jest całkowalna według dwóch różnych definicji całki oznaczonej, wynik całkowania będzie taki sam. Jeśli jak poprzednio przez $F(x)$ oznaczymy funkcję pierwotną $f(x)$, to całkę oznaczoną na przedziale $[a, b]$ obliczymy w następujący sposób:
$$ \int \limits _{a}^{b}f(x)dx = F(b) - F(a)$$
	
\subsection{Całkowanie górne i dolne}
\begin{definition}
	Podziałem $P = (x_0, x_1, x_2, ..., x_n)$ odcinka $[a ,b]$ nazywamy dowolny, skończony rosnący ciąg liczb $(x_0, x_1, x_2, ..., x_n)$ taki, że $x_0 = a$, $x_n = b$, $n \in N_+$.
\end{definition}
\begin{definition}
	Niech $f(x)$ będzie ograniczoną funkcją na $[a, b]$. Sumą górną $f(x)$ względem $P$ nazywamy $U(f, P) = \sum_{i=1}^{n} \Delta x_i m_i$, gdzie $\Delta x_i = x_{i-1} - x_i$, $M_i = \sup \left\lbrace f(x) : x \in [x_{i-1}, x_i] \right\rbrace $
\end{definition}
Analogicznie definiujemy sumę dolną i oznaczamy: $L(f, P)$.
\begin{definition}
	Niech $f$ będzie funkcją ciągłą na przedziale $[a, b]$, P --- podział odcinka $[a, b]$. Całką górną nazywamy:
	$$\overline{\int_{a}^{{b}}} f = \inf \left\lbrace U(f, P) \right\rbrace $$
\end{definition}
Powyższą definicję interpretujemy jako najmniejszą sumę górnego podziału odcinka $[a, b]$.

\subsection{Całkowanie numeryczne: przez trapezy}
To całkowanie polega na przybliżaniu pola pod wykresem funkcji $f(x)$ poprzez podział odcinka $[a, b]$ na $n$ części i wyznaczenie pola trapezu, gdzie wysokość w punkcie $x_{i-1}$ jest równa $f(x_{i-1})$, a w punkcie $x_{i}$ jest równa $f(x_{i})$. W ten sposób dostajemy $n$ trapezów, których zsumowane pole przybliża faktyczne pole pod wykresem funkcji $f(x)$ - im większa wartość n, tym lepsze przybliżenie.

\section{Analiza zadanej funkcji}
Wybrana przez nas funkcja ma postać: $f(x) = \py{sympy.latex(funkcja)}$. Poniżej przedstawiamy wizualne reprezentacje różnych metod całkowania i osiągnięte wyniki.

\subsection{Całka Riemanna}
Wizualizacja graficzno-geometryczna całki w sensie Riemanna:
\begin{figure}[h]
	%% Creator: Matplotlib, PGF backend
%%
%% To include the figure in your LaTeX document, write
%%   \input{<filename>.pgf}
%%
%% Make sure the required packages are loaded in your preamble
%%   \usepackage{pgf}
%%
%% Also ensure that all the required font packages are loaded; for instance,
%% the lmodern package is sometimes necessary when using math font.
%%   \usepackage{lmodern}
%%
%% Figures using additional raster images can only be included by \input if
%% they are in the same directory as the main LaTeX file. For loading figures
%% from other directories you can use the `import` package
%%   \usepackage{import}
%%
%% and then include the figures with
%%   \import{<path to file>}{<filename>.pgf}
%%
%% Matplotlib used the following preamble
%%   
%%   \makeatletter\@ifpackageloaded{underscore}{}{\usepackage[strings]{underscore}}\makeatother
%%
\begingroup%
\makeatletter%
\begin{pgfpicture}%
\pgfpathrectangle{\pgfpointorigin}{\pgfqpoint{6.400000in}{4.800000in}}%
\pgfusepath{use as bounding box, clip}%
\begin{pgfscope}%
\pgfsetbuttcap%
\pgfsetmiterjoin%
\definecolor{currentfill}{rgb}{1.000000,1.000000,1.000000}%
\pgfsetfillcolor{currentfill}%
\pgfsetlinewidth{0.000000pt}%
\definecolor{currentstroke}{rgb}{1.000000,1.000000,1.000000}%
\pgfsetstrokecolor{currentstroke}%
\pgfsetdash{}{0pt}%
\pgfpathmoveto{\pgfqpoint{0.000000in}{0.000000in}}%
\pgfpathlineto{\pgfqpoint{6.400000in}{0.000000in}}%
\pgfpathlineto{\pgfqpoint{6.400000in}{4.800000in}}%
\pgfpathlineto{\pgfqpoint{0.000000in}{4.800000in}}%
\pgfpathlineto{\pgfqpoint{0.000000in}{0.000000in}}%
\pgfpathclose%
\pgfusepath{fill}%
\end{pgfscope}%
\begin{pgfscope}%
\pgfsetbuttcap%
\pgfsetmiterjoin%
\definecolor{currentfill}{rgb}{1.000000,1.000000,1.000000}%
\pgfsetfillcolor{currentfill}%
\pgfsetlinewidth{0.000000pt}%
\definecolor{currentstroke}{rgb}{0.000000,0.000000,0.000000}%
\pgfsetstrokecolor{currentstroke}%
\pgfsetstrokeopacity{0.000000}%
\pgfsetdash{}{0pt}%
\pgfpathmoveto{\pgfqpoint{0.800000in}{0.528000in}}%
\pgfpathlineto{\pgfqpoint{5.760000in}{0.528000in}}%
\pgfpathlineto{\pgfqpoint{5.760000in}{4.224000in}}%
\pgfpathlineto{\pgfqpoint{0.800000in}{4.224000in}}%
\pgfpathlineto{\pgfqpoint{0.800000in}{0.528000in}}%
\pgfpathclose%
\pgfusepath{fill}%
\end{pgfscope}%
\begin{pgfscope}%
\pgfpathrectangle{\pgfqpoint{0.800000in}{0.528000in}}{\pgfqpoint{4.960000in}{3.696000in}}%
\pgfusepath{clip}%
\pgfsetbuttcap%
\pgfsetroundjoin%
\definecolor{currentfill}{rgb}{0.121569,0.466667,0.705882}%
\pgfsetfillcolor{currentfill}%
\pgfsetfillopacity{0.300000}%
\pgfsetlinewidth{0.000000pt}%
\definecolor{currentstroke}{rgb}{0.000000,0.000000,0.000000}%
\pgfsetstrokecolor{currentstroke}%
\pgfsetdash{}{0pt}%
\pgfpathmoveto{\pgfqpoint{1.055717in}{0.810962in}}%
\pgfpathlineto{\pgfqpoint{1.055717in}{0.697572in}}%
\pgfpathlineto{\pgfqpoint{1.085979in}{0.699743in}}%
\pgfpathlineto{\pgfqpoint{1.116242in}{0.702466in}}%
\pgfpathlineto{\pgfqpoint{1.146504in}{0.705708in}}%
\pgfpathlineto{\pgfqpoint{1.176766in}{0.709442in}}%
\pgfpathlineto{\pgfqpoint{1.207029in}{0.713648in}}%
\pgfpathlineto{\pgfqpoint{1.237291in}{0.718309in}}%
\pgfpathlineto{\pgfqpoint{1.267553in}{0.723412in}}%
\pgfpathlineto{\pgfqpoint{1.297816in}{0.728947in}}%
\pgfpathlineto{\pgfqpoint{1.328078in}{0.734905in}}%
\pgfpathlineto{\pgfqpoint{1.358340in}{0.741277in}}%
\pgfpathlineto{\pgfqpoint{1.388603in}{0.748057in}}%
\pgfpathlineto{\pgfqpoint{1.418865in}{0.755239in}}%
\pgfpathlineto{\pgfqpoint{1.449128in}{0.762816in}}%
\pgfpathlineto{\pgfqpoint{1.479390in}{0.770783in}}%
\pgfpathlineto{\pgfqpoint{1.509652in}{0.779134in}}%
\pgfpathlineto{\pgfqpoint{1.539915in}{0.787865in}}%
\pgfpathlineto{\pgfqpoint{1.570177in}{0.796969in}}%
\pgfpathlineto{\pgfqpoint{1.600439in}{0.806442in}}%
\pgfpathlineto{\pgfqpoint{1.630702in}{0.816277in}}%
\pgfpathlineto{\pgfqpoint{1.660964in}{0.826469in}}%
\pgfpathlineto{\pgfqpoint{1.691226in}{0.837012in}}%
\pgfpathlineto{\pgfqpoint{1.721489in}{0.847900in}}%
\pgfpathlineto{\pgfqpoint{1.751751in}{0.859126in}}%
\pgfpathlineto{\pgfqpoint{1.782013in}{0.870684in}}%
\pgfpathlineto{\pgfqpoint{1.812276in}{0.882568in}}%
\pgfpathlineto{\pgfqpoint{1.842538in}{0.894771in}}%
\pgfpathlineto{\pgfqpoint{1.872800in}{0.907285in}}%
\pgfpathlineto{\pgfqpoint{1.903063in}{0.920103in}}%
\pgfpathlineto{\pgfqpoint{1.933325in}{0.933219in}}%
\pgfpathlineto{\pgfqpoint{1.963588in}{0.946625in}}%
\pgfpathlineto{\pgfqpoint{1.993850in}{0.960313in}}%
\pgfpathlineto{\pgfqpoint{2.024112in}{0.974276in}}%
\pgfpathlineto{\pgfqpoint{2.054375in}{0.988506in}}%
\pgfpathlineto{\pgfqpoint{2.084637in}{1.002996in}}%
\pgfpathlineto{\pgfqpoint{2.114899in}{1.017738in}}%
\pgfpathlineto{\pgfqpoint{2.145162in}{1.032724in}}%
\pgfpathlineto{\pgfqpoint{2.175424in}{1.047946in}}%
\pgfpathlineto{\pgfqpoint{2.205686in}{1.063398in}}%
\pgfpathlineto{\pgfqpoint{2.235949in}{1.079072in}}%
\pgfpathlineto{\pgfqpoint{2.266211in}{1.094959in}}%
\pgfpathlineto{\pgfqpoint{2.296473in}{1.111054in}}%
\pgfpathlineto{\pgfqpoint{2.326736in}{1.127349in}}%
\pgfpathlineto{\pgfqpoint{2.356998in}{1.143837in}}%
\pgfpathlineto{\pgfqpoint{2.387261in}{1.160512in}}%
\pgfpathlineto{\pgfqpoint{2.417523in}{1.177366in}}%
\pgfpathlineto{\pgfqpoint{2.447785in}{1.194394in}}%
\pgfpathlineto{\pgfqpoint{2.478048in}{1.211591in}}%
\pgfpathlineto{\pgfqpoint{2.508310in}{1.228950in}}%
\pgfpathlineto{\pgfqpoint{2.538572in}{1.246465in}}%
\pgfpathlineto{\pgfqpoint{2.568835in}{1.264133in}}%
\pgfpathlineto{\pgfqpoint{2.599097in}{1.281948in}}%
\pgfpathlineto{\pgfqpoint{2.629359in}{1.299907in}}%
\pgfpathlineto{\pgfqpoint{2.659622in}{1.318005in}}%
\pgfpathlineto{\pgfqpoint{2.689884in}{1.336238in}}%
\pgfpathlineto{\pgfqpoint{2.720146in}{1.354605in}}%
\pgfpathlineto{\pgfqpoint{2.750409in}{1.373101in}}%
\pgfpathlineto{\pgfqpoint{2.780671in}{1.391725in}}%
\pgfpathlineto{\pgfqpoint{2.810933in}{1.410476in}}%
\pgfpathlineto{\pgfqpoint{2.841196in}{1.429351in}}%
\pgfpathlineto{\pgfqpoint{2.871458in}{1.448350in}}%
\pgfpathlineto{\pgfqpoint{2.901721in}{1.467473in}}%
\pgfpathlineto{\pgfqpoint{2.931983in}{1.486718in}}%
\pgfpathlineto{\pgfqpoint{2.962245in}{1.506088in}}%
\pgfpathlineto{\pgfqpoint{2.992508in}{1.525582in}}%
\pgfpathlineto{\pgfqpoint{3.022770in}{1.545202in}}%
\pgfpathlineto{\pgfqpoint{3.053032in}{1.564950in}}%
\pgfpathlineto{\pgfqpoint{3.083295in}{1.584828in}}%
\pgfpathlineto{\pgfqpoint{3.113557in}{1.604838in}}%
\pgfpathlineto{\pgfqpoint{3.143819in}{1.624984in}}%
\pgfpathlineto{\pgfqpoint{3.174082in}{1.645269in}}%
\pgfpathlineto{\pgfqpoint{3.204344in}{1.665697in}}%
\pgfpathlineto{\pgfqpoint{3.234606in}{1.686272in}}%
\pgfpathlineto{\pgfqpoint{3.264869in}{1.706998in}}%
\pgfpathlineto{\pgfqpoint{3.295131in}{1.727882in}}%
\pgfpathlineto{\pgfqpoint{3.325394in}{1.748926in}}%
\pgfpathlineto{\pgfqpoint{3.355656in}{1.770138in}}%
\pgfpathlineto{\pgfqpoint{3.385918in}{1.791524in}}%
\pgfpathlineto{\pgfqpoint{3.416181in}{1.813088in}}%
\pgfpathlineto{\pgfqpoint{3.446443in}{1.834838in}}%
\pgfpathlineto{\pgfqpoint{3.476705in}{1.856780in}}%
\pgfpathlineto{\pgfqpoint{3.506968in}{1.878921in}}%
\pgfpathlineto{\pgfqpoint{3.537230in}{1.901268in}}%
\pgfpathlineto{\pgfqpoint{3.567492in}{1.923827in}}%
\pgfpathlineto{\pgfqpoint{3.597755in}{1.946607in}}%
\pgfpathlineto{\pgfqpoint{3.628017in}{1.969614in}}%
\pgfpathlineto{\pgfqpoint{3.658279in}{1.992855in}}%
\pgfpathlineto{\pgfqpoint{3.688542in}{2.016338in}}%
\pgfpathlineto{\pgfqpoint{3.718804in}{2.040071in}}%
\pgfpathlineto{\pgfqpoint{3.749067in}{2.064061in}}%
\pgfpathlineto{\pgfqpoint{3.779329in}{2.088314in}}%
\pgfpathlineto{\pgfqpoint{3.809591in}{2.112840in}}%
\pgfpathlineto{\pgfqpoint{3.839854in}{2.137643in}}%
\pgfpathlineto{\pgfqpoint{3.870116in}{2.162733in}}%
\pgfpathlineto{\pgfqpoint{3.900378in}{2.188115in}}%
\pgfpathlineto{\pgfqpoint{3.930641in}{2.213796in}}%
\pgfpathlineto{\pgfqpoint{3.960903in}{2.239783in}}%
\pgfpathlineto{\pgfqpoint{3.991165in}{2.266082in}}%
\pgfpathlineto{\pgfqpoint{4.021428in}{2.292699in}}%
\pgfpathlineto{\pgfqpoint{4.051690in}{2.319641in}}%
\pgfpathlineto{\pgfqpoint{4.081952in}{2.346912in}}%
\pgfpathlineto{\pgfqpoint{4.112215in}{2.374517in}}%
\pgfpathlineto{\pgfqpoint{4.142477in}{2.402463in}}%
\pgfpathlineto{\pgfqpoint{4.172739in}{2.430753in}}%
\pgfpathlineto{\pgfqpoint{4.203002in}{2.459391in}}%
\pgfpathlineto{\pgfqpoint{4.233264in}{2.488381in}}%
\pgfpathlineto{\pgfqpoint{4.263527in}{2.517727in}}%
\pgfpathlineto{\pgfqpoint{4.293789in}{2.547432in}}%
\pgfpathlineto{\pgfqpoint{4.324051in}{2.577498in}}%
\pgfpathlineto{\pgfqpoint{4.354314in}{2.607928in}}%
\pgfpathlineto{\pgfqpoint{4.384576in}{2.638723in}}%
\pgfpathlineto{\pgfqpoint{4.414838in}{2.669884in}}%
\pgfpathlineto{\pgfqpoint{4.445101in}{2.701413in}}%
\pgfpathlineto{\pgfqpoint{4.475363in}{2.733310in}}%
\pgfpathlineto{\pgfqpoint{4.505625in}{2.765574in}}%
\pgfpathlineto{\pgfqpoint{4.535888in}{2.798206in}}%
\pgfpathlineto{\pgfqpoint{4.566150in}{2.831205in}}%
\pgfpathlineto{\pgfqpoint{4.596412in}{2.864569in}}%
\pgfpathlineto{\pgfqpoint{4.626675in}{2.898296in}}%
\pgfpathlineto{\pgfqpoint{4.656937in}{2.932384in}}%
\pgfpathlineto{\pgfqpoint{4.687200in}{2.966830in}}%
\pgfpathlineto{\pgfqpoint{4.717462in}{3.001633in}}%
\pgfpathlineto{\pgfqpoint{4.747724in}{3.036787in}}%
\pgfpathlineto{\pgfqpoint{4.777987in}{3.072289in}}%
\pgfpathlineto{\pgfqpoint{4.808249in}{3.108135in}}%
\pgfpathlineto{\pgfqpoint{4.838511in}{3.144321in}}%
\pgfpathlineto{\pgfqpoint{4.868774in}{3.180841in}}%
\pgfpathlineto{\pgfqpoint{4.899036in}{3.217690in}}%
\pgfpathlineto{\pgfqpoint{4.929298in}{3.254863in}}%
\pgfpathlineto{\pgfqpoint{4.959561in}{3.292353in}}%
\pgfpathlineto{\pgfqpoint{4.989823in}{3.330156in}}%
\pgfpathlineto{\pgfqpoint{5.020085in}{3.368264in}}%
\pgfpathlineto{\pgfqpoint{5.050348in}{3.406670in}}%
\pgfpathlineto{\pgfqpoint{5.080610in}{3.445369in}}%
\pgfpathlineto{\pgfqpoint{5.110872in}{3.484352in}}%
\pgfpathlineto{\pgfqpoint{5.141135in}{3.523614in}}%
\pgfpathlineto{\pgfqpoint{5.171397in}{3.563147in}}%
\pgfpathlineto{\pgfqpoint{5.201660in}{3.602943in}}%
\pgfpathlineto{\pgfqpoint{5.231922in}{3.642995in}}%
\pgfpathlineto{\pgfqpoint{5.262184in}{3.683297in}}%
\pgfpathlineto{\pgfqpoint{5.292447in}{3.723839in}}%
\pgfpathlineto{\pgfqpoint{5.322709in}{3.764616in}}%
\pgfpathlineto{\pgfqpoint{5.352971in}{3.805619in}}%
\pgfpathlineto{\pgfqpoint{5.383234in}{3.846842in}}%
\pgfpathlineto{\pgfqpoint{5.413496in}{3.888277in}}%
\pgfpathlineto{\pgfqpoint{5.443758in}{3.929917in}}%
\pgfpathlineto{\pgfqpoint{5.474021in}{3.971755in}}%
\pgfpathlineto{\pgfqpoint{5.504283in}{4.013785in}}%
\pgfpathlineto{\pgfqpoint{5.504283in}{0.810962in}}%
\pgfpathlineto{\pgfqpoint{5.504283in}{0.810962in}}%
\pgfpathlineto{\pgfqpoint{5.474021in}{0.810962in}}%
\pgfpathlineto{\pgfqpoint{5.443758in}{0.810962in}}%
\pgfpathlineto{\pgfqpoint{5.413496in}{0.810962in}}%
\pgfpathlineto{\pgfqpoint{5.383234in}{0.810962in}}%
\pgfpathlineto{\pgfqpoint{5.352971in}{0.810962in}}%
\pgfpathlineto{\pgfqpoint{5.322709in}{0.810962in}}%
\pgfpathlineto{\pgfqpoint{5.292447in}{0.810962in}}%
\pgfpathlineto{\pgfqpoint{5.262184in}{0.810962in}}%
\pgfpathlineto{\pgfqpoint{5.231922in}{0.810962in}}%
\pgfpathlineto{\pgfqpoint{5.201660in}{0.810962in}}%
\pgfpathlineto{\pgfqpoint{5.171397in}{0.810962in}}%
\pgfpathlineto{\pgfqpoint{5.141135in}{0.810962in}}%
\pgfpathlineto{\pgfqpoint{5.110872in}{0.810962in}}%
\pgfpathlineto{\pgfqpoint{5.080610in}{0.810962in}}%
\pgfpathlineto{\pgfqpoint{5.050348in}{0.810962in}}%
\pgfpathlineto{\pgfqpoint{5.020085in}{0.810962in}}%
\pgfpathlineto{\pgfqpoint{4.989823in}{0.810962in}}%
\pgfpathlineto{\pgfqpoint{4.959561in}{0.810962in}}%
\pgfpathlineto{\pgfqpoint{4.929298in}{0.810962in}}%
\pgfpathlineto{\pgfqpoint{4.899036in}{0.810962in}}%
\pgfpathlineto{\pgfqpoint{4.868774in}{0.810962in}}%
\pgfpathlineto{\pgfqpoint{4.838511in}{0.810962in}}%
\pgfpathlineto{\pgfqpoint{4.808249in}{0.810962in}}%
\pgfpathlineto{\pgfqpoint{4.777987in}{0.810962in}}%
\pgfpathlineto{\pgfqpoint{4.747724in}{0.810962in}}%
\pgfpathlineto{\pgfqpoint{4.717462in}{0.810962in}}%
\pgfpathlineto{\pgfqpoint{4.687200in}{0.810962in}}%
\pgfpathlineto{\pgfqpoint{4.656937in}{0.810962in}}%
\pgfpathlineto{\pgfqpoint{4.626675in}{0.810962in}}%
\pgfpathlineto{\pgfqpoint{4.596412in}{0.810962in}}%
\pgfpathlineto{\pgfqpoint{4.566150in}{0.810962in}}%
\pgfpathlineto{\pgfqpoint{4.535888in}{0.810962in}}%
\pgfpathlineto{\pgfqpoint{4.505625in}{0.810962in}}%
\pgfpathlineto{\pgfqpoint{4.475363in}{0.810962in}}%
\pgfpathlineto{\pgfqpoint{4.445101in}{0.810962in}}%
\pgfpathlineto{\pgfqpoint{4.414838in}{0.810962in}}%
\pgfpathlineto{\pgfqpoint{4.384576in}{0.810962in}}%
\pgfpathlineto{\pgfqpoint{4.354314in}{0.810962in}}%
\pgfpathlineto{\pgfqpoint{4.324051in}{0.810962in}}%
\pgfpathlineto{\pgfqpoint{4.293789in}{0.810962in}}%
\pgfpathlineto{\pgfqpoint{4.263527in}{0.810962in}}%
\pgfpathlineto{\pgfqpoint{4.233264in}{0.810962in}}%
\pgfpathlineto{\pgfqpoint{4.203002in}{0.810962in}}%
\pgfpathlineto{\pgfqpoint{4.172739in}{0.810962in}}%
\pgfpathlineto{\pgfqpoint{4.142477in}{0.810962in}}%
\pgfpathlineto{\pgfqpoint{4.112215in}{0.810962in}}%
\pgfpathlineto{\pgfqpoint{4.081952in}{0.810962in}}%
\pgfpathlineto{\pgfqpoint{4.051690in}{0.810962in}}%
\pgfpathlineto{\pgfqpoint{4.021428in}{0.810962in}}%
\pgfpathlineto{\pgfqpoint{3.991165in}{0.810962in}}%
\pgfpathlineto{\pgfqpoint{3.960903in}{0.810962in}}%
\pgfpathlineto{\pgfqpoint{3.930641in}{0.810962in}}%
\pgfpathlineto{\pgfqpoint{3.900378in}{0.810962in}}%
\pgfpathlineto{\pgfqpoint{3.870116in}{0.810962in}}%
\pgfpathlineto{\pgfqpoint{3.839854in}{0.810962in}}%
\pgfpathlineto{\pgfqpoint{3.809591in}{0.810962in}}%
\pgfpathlineto{\pgfqpoint{3.779329in}{0.810962in}}%
\pgfpathlineto{\pgfqpoint{3.749067in}{0.810962in}}%
\pgfpathlineto{\pgfqpoint{3.718804in}{0.810962in}}%
\pgfpathlineto{\pgfqpoint{3.688542in}{0.810962in}}%
\pgfpathlineto{\pgfqpoint{3.658279in}{0.810962in}}%
\pgfpathlineto{\pgfqpoint{3.628017in}{0.810962in}}%
\pgfpathlineto{\pgfqpoint{3.597755in}{0.810962in}}%
\pgfpathlineto{\pgfqpoint{3.567492in}{0.810962in}}%
\pgfpathlineto{\pgfqpoint{3.537230in}{0.810962in}}%
\pgfpathlineto{\pgfqpoint{3.506968in}{0.810962in}}%
\pgfpathlineto{\pgfqpoint{3.476705in}{0.810962in}}%
\pgfpathlineto{\pgfqpoint{3.446443in}{0.810962in}}%
\pgfpathlineto{\pgfqpoint{3.416181in}{0.810962in}}%
\pgfpathlineto{\pgfqpoint{3.385918in}{0.810962in}}%
\pgfpathlineto{\pgfqpoint{3.355656in}{0.810962in}}%
\pgfpathlineto{\pgfqpoint{3.325394in}{0.810962in}}%
\pgfpathlineto{\pgfqpoint{3.295131in}{0.810962in}}%
\pgfpathlineto{\pgfqpoint{3.264869in}{0.810962in}}%
\pgfpathlineto{\pgfqpoint{3.234606in}{0.810962in}}%
\pgfpathlineto{\pgfqpoint{3.204344in}{0.810962in}}%
\pgfpathlineto{\pgfqpoint{3.174082in}{0.810962in}}%
\pgfpathlineto{\pgfqpoint{3.143819in}{0.810962in}}%
\pgfpathlineto{\pgfqpoint{3.113557in}{0.810962in}}%
\pgfpathlineto{\pgfqpoint{3.083295in}{0.810962in}}%
\pgfpathlineto{\pgfqpoint{3.053032in}{0.810962in}}%
\pgfpathlineto{\pgfqpoint{3.022770in}{0.810962in}}%
\pgfpathlineto{\pgfqpoint{2.992508in}{0.810962in}}%
\pgfpathlineto{\pgfqpoint{2.962245in}{0.810962in}}%
\pgfpathlineto{\pgfqpoint{2.931983in}{0.810962in}}%
\pgfpathlineto{\pgfqpoint{2.901721in}{0.810962in}}%
\pgfpathlineto{\pgfqpoint{2.871458in}{0.810962in}}%
\pgfpathlineto{\pgfqpoint{2.841196in}{0.810962in}}%
\pgfpathlineto{\pgfqpoint{2.810933in}{0.810962in}}%
\pgfpathlineto{\pgfqpoint{2.780671in}{0.810962in}}%
\pgfpathlineto{\pgfqpoint{2.750409in}{0.810962in}}%
\pgfpathlineto{\pgfqpoint{2.720146in}{0.810962in}}%
\pgfpathlineto{\pgfqpoint{2.689884in}{0.810962in}}%
\pgfpathlineto{\pgfqpoint{2.659622in}{0.810962in}}%
\pgfpathlineto{\pgfqpoint{2.629359in}{0.810962in}}%
\pgfpathlineto{\pgfqpoint{2.599097in}{0.810962in}}%
\pgfpathlineto{\pgfqpoint{2.568835in}{0.810962in}}%
\pgfpathlineto{\pgfqpoint{2.538572in}{0.810962in}}%
\pgfpathlineto{\pgfqpoint{2.508310in}{0.810962in}}%
\pgfpathlineto{\pgfqpoint{2.478048in}{0.810962in}}%
\pgfpathlineto{\pgfqpoint{2.447785in}{0.810962in}}%
\pgfpathlineto{\pgfqpoint{2.417523in}{0.810962in}}%
\pgfpathlineto{\pgfqpoint{2.387261in}{0.810962in}}%
\pgfpathlineto{\pgfqpoint{2.356998in}{0.810962in}}%
\pgfpathlineto{\pgfqpoint{2.326736in}{0.810962in}}%
\pgfpathlineto{\pgfqpoint{2.296473in}{0.810962in}}%
\pgfpathlineto{\pgfqpoint{2.266211in}{0.810962in}}%
\pgfpathlineto{\pgfqpoint{2.235949in}{0.810962in}}%
\pgfpathlineto{\pgfqpoint{2.205686in}{0.810962in}}%
\pgfpathlineto{\pgfqpoint{2.175424in}{0.810962in}}%
\pgfpathlineto{\pgfqpoint{2.145162in}{0.810962in}}%
\pgfpathlineto{\pgfqpoint{2.114899in}{0.810962in}}%
\pgfpathlineto{\pgfqpoint{2.084637in}{0.810962in}}%
\pgfpathlineto{\pgfqpoint{2.054375in}{0.810962in}}%
\pgfpathlineto{\pgfqpoint{2.024112in}{0.810962in}}%
\pgfpathlineto{\pgfqpoint{1.993850in}{0.810962in}}%
\pgfpathlineto{\pgfqpoint{1.963588in}{0.810962in}}%
\pgfpathlineto{\pgfqpoint{1.933325in}{0.810962in}}%
\pgfpathlineto{\pgfqpoint{1.903063in}{0.810962in}}%
\pgfpathlineto{\pgfqpoint{1.872800in}{0.810962in}}%
\pgfpathlineto{\pgfqpoint{1.842538in}{0.810962in}}%
\pgfpathlineto{\pgfqpoint{1.812276in}{0.810962in}}%
\pgfpathlineto{\pgfqpoint{1.782013in}{0.810962in}}%
\pgfpathlineto{\pgfqpoint{1.751751in}{0.810962in}}%
\pgfpathlineto{\pgfqpoint{1.721489in}{0.810962in}}%
\pgfpathlineto{\pgfqpoint{1.691226in}{0.810962in}}%
\pgfpathlineto{\pgfqpoint{1.660964in}{0.810962in}}%
\pgfpathlineto{\pgfqpoint{1.630702in}{0.810962in}}%
\pgfpathlineto{\pgfqpoint{1.600439in}{0.810962in}}%
\pgfpathlineto{\pgfqpoint{1.570177in}{0.810962in}}%
\pgfpathlineto{\pgfqpoint{1.539915in}{0.810962in}}%
\pgfpathlineto{\pgfqpoint{1.509652in}{0.810962in}}%
\pgfpathlineto{\pgfqpoint{1.479390in}{0.810962in}}%
\pgfpathlineto{\pgfqpoint{1.449128in}{0.810962in}}%
\pgfpathlineto{\pgfqpoint{1.418865in}{0.810962in}}%
\pgfpathlineto{\pgfqpoint{1.388603in}{0.810962in}}%
\pgfpathlineto{\pgfqpoint{1.358340in}{0.810962in}}%
\pgfpathlineto{\pgfqpoint{1.328078in}{0.810962in}}%
\pgfpathlineto{\pgfqpoint{1.297816in}{0.810962in}}%
\pgfpathlineto{\pgfqpoint{1.267553in}{0.810962in}}%
\pgfpathlineto{\pgfqpoint{1.237291in}{0.810962in}}%
\pgfpathlineto{\pgfqpoint{1.207029in}{0.810962in}}%
\pgfpathlineto{\pgfqpoint{1.176766in}{0.810962in}}%
\pgfpathlineto{\pgfqpoint{1.146504in}{0.810962in}}%
\pgfpathlineto{\pgfqpoint{1.116242in}{0.810962in}}%
\pgfpathlineto{\pgfqpoint{1.085979in}{0.810962in}}%
\pgfpathlineto{\pgfqpoint{1.055717in}{0.810962in}}%
\pgfpathlineto{\pgfqpoint{1.055717in}{0.810962in}}%
\pgfpathclose%
\pgfusepath{fill}%
\end{pgfscope}%
\begin{pgfscope}%
\pgfsetbuttcap%
\pgfsetroundjoin%
\definecolor{currentfill}{rgb}{0.000000,0.000000,0.000000}%
\pgfsetfillcolor{currentfill}%
\pgfsetlinewidth{0.803000pt}%
\definecolor{currentstroke}{rgb}{0.000000,0.000000,0.000000}%
\pgfsetstrokecolor{currentstroke}%
\pgfsetdash{}{0pt}%
\pgfsys@defobject{currentmarker}{\pgfqpoint{0.000000in}{-0.048611in}}{\pgfqpoint{0.000000in}{0.000000in}}{%
\pgfpathmoveto{\pgfqpoint{0.000000in}{0.000000in}}%
\pgfpathlineto{\pgfqpoint{0.000000in}{-0.048611in}}%
\pgfusepath{stroke,fill}%
}%
\begin{pgfscope}%
\pgfsys@transformshift{1.526465in}{0.528000in}%
\pgfsys@useobject{currentmarker}{}%
\end{pgfscope}%
\end{pgfscope}%
\begin{pgfscope}%
\definecolor{textcolor}{rgb}{0.000000,0.000000,0.000000}%
\pgfsetstrokecolor{textcolor}%
\pgfsetfillcolor{textcolor}%
\pgftext[x=1.526465in,y=0.430778in,,top]{\color{textcolor}\rmfamily\fontsize{10.000000}{12.000000}\selectfont \(\displaystyle {2}\)}%
\end{pgfscope}%
\begin{pgfscope}%
\pgfsetbuttcap%
\pgfsetroundjoin%
\definecolor{currentfill}{rgb}{0.000000,0.000000,0.000000}%
\pgfsetfillcolor{currentfill}%
\pgfsetlinewidth{0.803000pt}%
\definecolor{currentstroke}{rgb}{0.000000,0.000000,0.000000}%
\pgfsetstrokecolor{currentstroke}%
\pgfsetdash{}{0pt}%
\pgfsys@defobject{currentmarker}{\pgfqpoint{0.000000in}{-0.048611in}}{\pgfqpoint{0.000000in}{0.000000in}}{%
\pgfpathmoveto{\pgfqpoint{0.000000in}{0.000000in}}%
\pgfpathlineto{\pgfqpoint{0.000000in}{-0.048611in}}%
\pgfusepath{stroke,fill}%
}%
\begin{pgfscope}%
\pgfsys@transformshift{2.528485in}{0.528000in}%
\pgfsys@useobject{currentmarker}{}%
\end{pgfscope}%
\end{pgfscope}%
\begin{pgfscope}%
\definecolor{textcolor}{rgb}{0.000000,0.000000,0.000000}%
\pgfsetstrokecolor{textcolor}%
\pgfsetfillcolor{textcolor}%
\pgftext[x=2.528485in,y=0.430778in,,top]{\color{textcolor}\rmfamily\fontsize{10.000000}{12.000000}\selectfont \(\displaystyle {4}\)}%
\end{pgfscope}%
\begin{pgfscope}%
\pgfsetbuttcap%
\pgfsetroundjoin%
\definecolor{currentfill}{rgb}{0.000000,0.000000,0.000000}%
\pgfsetfillcolor{currentfill}%
\pgfsetlinewidth{0.803000pt}%
\definecolor{currentstroke}{rgb}{0.000000,0.000000,0.000000}%
\pgfsetstrokecolor{currentstroke}%
\pgfsetdash{}{0pt}%
\pgfsys@defobject{currentmarker}{\pgfqpoint{0.000000in}{-0.048611in}}{\pgfqpoint{0.000000in}{0.000000in}}{%
\pgfpathmoveto{\pgfqpoint{0.000000in}{0.000000in}}%
\pgfpathlineto{\pgfqpoint{0.000000in}{-0.048611in}}%
\pgfusepath{stroke,fill}%
}%
\begin{pgfscope}%
\pgfsys@transformshift{3.530505in}{0.528000in}%
\pgfsys@useobject{currentmarker}{}%
\end{pgfscope}%
\end{pgfscope}%
\begin{pgfscope}%
\definecolor{textcolor}{rgb}{0.000000,0.000000,0.000000}%
\pgfsetstrokecolor{textcolor}%
\pgfsetfillcolor{textcolor}%
\pgftext[x=3.530505in,y=0.430778in,,top]{\color{textcolor}\rmfamily\fontsize{10.000000}{12.000000}\selectfont \(\displaystyle {6}\)}%
\end{pgfscope}%
\begin{pgfscope}%
\pgfsetbuttcap%
\pgfsetroundjoin%
\definecolor{currentfill}{rgb}{0.000000,0.000000,0.000000}%
\pgfsetfillcolor{currentfill}%
\pgfsetlinewidth{0.803000pt}%
\definecolor{currentstroke}{rgb}{0.000000,0.000000,0.000000}%
\pgfsetstrokecolor{currentstroke}%
\pgfsetdash{}{0pt}%
\pgfsys@defobject{currentmarker}{\pgfqpoint{0.000000in}{-0.048611in}}{\pgfqpoint{0.000000in}{0.000000in}}{%
\pgfpathmoveto{\pgfqpoint{0.000000in}{0.000000in}}%
\pgfpathlineto{\pgfqpoint{0.000000in}{-0.048611in}}%
\pgfusepath{stroke,fill}%
}%
\begin{pgfscope}%
\pgfsys@transformshift{4.532525in}{0.528000in}%
\pgfsys@useobject{currentmarker}{}%
\end{pgfscope}%
\end{pgfscope}%
\begin{pgfscope}%
\definecolor{textcolor}{rgb}{0.000000,0.000000,0.000000}%
\pgfsetstrokecolor{textcolor}%
\pgfsetfillcolor{textcolor}%
\pgftext[x=4.532525in,y=0.430778in,,top]{\color{textcolor}\rmfamily\fontsize{10.000000}{12.000000}\selectfont \(\displaystyle {8}\)}%
\end{pgfscope}%
\begin{pgfscope}%
\pgfsetbuttcap%
\pgfsetroundjoin%
\definecolor{currentfill}{rgb}{0.000000,0.000000,0.000000}%
\pgfsetfillcolor{currentfill}%
\pgfsetlinewidth{0.803000pt}%
\definecolor{currentstroke}{rgb}{0.000000,0.000000,0.000000}%
\pgfsetstrokecolor{currentstroke}%
\pgfsetdash{}{0pt}%
\pgfsys@defobject{currentmarker}{\pgfqpoint{0.000000in}{-0.048611in}}{\pgfqpoint{0.000000in}{0.000000in}}{%
\pgfpathmoveto{\pgfqpoint{0.000000in}{0.000000in}}%
\pgfpathlineto{\pgfqpoint{0.000000in}{-0.048611in}}%
\pgfusepath{stroke,fill}%
}%
\begin{pgfscope}%
\pgfsys@transformshift{5.534545in}{0.528000in}%
\pgfsys@useobject{currentmarker}{}%
\end{pgfscope}%
\end{pgfscope}%
\begin{pgfscope}%
\definecolor{textcolor}{rgb}{0.000000,0.000000,0.000000}%
\pgfsetstrokecolor{textcolor}%
\pgfsetfillcolor{textcolor}%
\pgftext[x=5.534545in,y=0.430778in,,top]{\color{textcolor}\rmfamily\fontsize{10.000000}{12.000000}\selectfont \(\displaystyle {10}\)}%
\end{pgfscope}%
\begin{pgfscope}%
\pgfsetbuttcap%
\pgfsetroundjoin%
\definecolor{currentfill}{rgb}{0.000000,0.000000,0.000000}%
\pgfsetfillcolor{currentfill}%
\pgfsetlinewidth{0.803000pt}%
\definecolor{currentstroke}{rgb}{0.000000,0.000000,0.000000}%
\pgfsetstrokecolor{currentstroke}%
\pgfsetdash{}{0pt}%
\pgfsys@defobject{currentmarker}{\pgfqpoint{-0.048611in}{0.000000in}}{\pgfqpoint{-0.000000in}{0.000000in}}{%
\pgfpathmoveto{\pgfqpoint{-0.000000in}{0.000000in}}%
\pgfpathlineto{\pgfqpoint{-0.048611in}{0.000000in}}%
\pgfusepath{stroke,fill}%
}%
\begin{pgfscope}%
\pgfsys@transformshift{0.800000in}{0.810962in}%
\pgfsys@useobject{currentmarker}{}%
\end{pgfscope}%
\end{pgfscope}%
\begin{pgfscope}%
\definecolor{textcolor}{rgb}{0.000000,0.000000,0.000000}%
\pgfsetstrokecolor{textcolor}%
\pgfsetfillcolor{textcolor}%
\pgftext[x=0.633333in, y=0.762737in, left, base]{\color{textcolor}\rmfamily\fontsize{10.000000}{12.000000}\selectfont \(\displaystyle {0}\)}%
\end{pgfscope}%
\begin{pgfscope}%
\pgfsetbuttcap%
\pgfsetroundjoin%
\definecolor{currentfill}{rgb}{0.000000,0.000000,0.000000}%
\pgfsetfillcolor{currentfill}%
\pgfsetlinewidth{0.803000pt}%
\definecolor{currentstroke}{rgb}{0.000000,0.000000,0.000000}%
\pgfsetstrokecolor{currentstroke}%
\pgfsetdash{}{0pt}%
\pgfsys@defobject{currentmarker}{\pgfqpoint{-0.048611in}{0.000000in}}{\pgfqpoint{-0.000000in}{0.000000in}}{%
\pgfpathmoveto{\pgfqpoint{-0.000000in}{0.000000in}}%
\pgfpathlineto{\pgfqpoint{-0.048611in}{0.000000in}}%
\pgfusepath{stroke,fill}%
}%
\begin{pgfscope}%
\pgfsys@transformshift{0.800000in}{1.483173in}%
\pgfsys@useobject{currentmarker}{}%
\end{pgfscope}%
\end{pgfscope}%
\begin{pgfscope}%
\definecolor{textcolor}{rgb}{0.000000,0.000000,0.000000}%
\pgfsetstrokecolor{textcolor}%
\pgfsetfillcolor{textcolor}%
\pgftext[x=0.563888in, y=1.434948in, left, base]{\color{textcolor}\rmfamily\fontsize{10.000000}{12.000000}\selectfont \(\displaystyle {20}\)}%
\end{pgfscope}%
\begin{pgfscope}%
\pgfsetbuttcap%
\pgfsetroundjoin%
\definecolor{currentfill}{rgb}{0.000000,0.000000,0.000000}%
\pgfsetfillcolor{currentfill}%
\pgfsetlinewidth{0.803000pt}%
\definecolor{currentstroke}{rgb}{0.000000,0.000000,0.000000}%
\pgfsetstrokecolor{currentstroke}%
\pgfsetdash{}{0pt}%
\pgfsys@defobject{currentmarker}{\pgfqpoint{-0.048611in}{0.000000in}}{\pgfqpoint{-0.000000in}{0.000000in}}{%
\pgfpathmoveto{\pgfqpoint{-0.000000in}{0.000000in}}%
\pgfpathlineto{\pgfqpoint{-0.048611in}{0.000000in}}%
\pgfusepath{stroke,fill}%
}%
\begin{pgfscope}%
\pgfsys@transformshift{0.800000in}{2.155384in}%
\pgfsys@useobject{currentmarker}{}%
\end{pgfscope}%
\end{pgfscope}%
\begin{pgfscope}%
\definecolor{textcolor}{rgb}{0.000000,0.000000,0.000000}%
\pgfsetstrokecolor{textcolor}%
\pgfsetfillcolor{textcolor}%
\pgftext[x=0.563888in, y=2.107158in, left, base]{\color{textcolor}\rmfamily\fontsize{10.000000}{12.000000}\selectfont \(\displaystyle {40}\)}%
\end{pgfscope}%
\begin{pgfscope}%
\pgfsetbuttcap%
\pgfsetroundjoin%
\definecolor{currentfill}{rgb}{0.000000,0.000000,0.000000}%
\pgfsetfillcolor{currentfill}%
\pgfsetlinewidth{0.803000pt}%
\definecolor{currentstroke}{rgb}{0.000000,0.000000,0.000000}%
\pgfsetstrokecolor{currentstroke}%
\pgfsetdash{}{0pt}%
\pgfsys@defobject{currentmarker}{\pgfqpoint{-0.048611in}{0.000000in}}{\pgfqpoint{-0.000000in}{0.000000in}}{%
\pgfpathmoveto{\pgfqpoint{-0.000000in}{0.000000in}}%
\pgfpathlineto{\pgfqpoint{-0.048611in}{0.000000in}}%
\pgfusepath{stroke,fill}%
}%
\begin{pgfscope}%
\pgfsys@transformshift{0.800000in}{2.827594in}%
\pgfsys@useobject{currentmarker}{}%
\end{pgfscope}%
\end{pgfscope}%
\begin{pgfscope}%
\definecolor{textcolor}{rgb}{0.000000,0.000000,0.000000}%
\pgfsetstrokecolor{textcolor}%
\pgfsetfillcolor{textcolor}%
\pgftext[x=0.563888in, y=2.779369in, left, base]{\color{textcolor}\rmfamily\fontsize{10.000000}{12.000000}\selectfont \(\displaystyle {60}\)}%
\end{pgfscope}%
\begin{pgfscope}%
\pgfsetbuttcap%
\pgfsetroundjoin%
\definecolor{currentfill}{rgb}{0.000000,0.000000,0.000000}%
\pgfsetfillcolor{currentfill}%
\pgfsetlinewidth{0.803000pt}%
\definecolor{currentstroke}{rgb}{0.000000,0.000000,0.000000}%
\pgfsetstrokecolor{currentstroke}%
\pgfsetdash{}{0pt}%
\pgfsys@defobject{currentmarker}{\pgfqpoint{-0.048611in}{0.000000in}}{\pgfqpoint{-0.000000in}{0.000000in}}{%
\pgfpathmoveto{\pgfqpoint{-0.000000in}{0.000000in}}%
\pgfpathlineto{\pgfqpoint{-0.048611in}{0.000000in}}%
\pgfusepath{stroke,fill}%
}%
\begin{pgfscope}%
\pgfsys@transformshift{0.800000in}{3.499805in}%
\pgfsys@useobject{currentmarker}{}%
\end{pgfscope}%
\end{pgfscope}%
\begin{pgfscope}%
\definecolor{textcolor}{rgb}{0.000000,0.000000,0.000000}%
\pgfsetstrokecolor{textcolor}%
\pgfsetfillcolor{textcolor}%
\pgftext[x=0.563888in, y=3.451580in, left, base]{\color{textcolor}\rmfamily\fontsize{10.000000}{12.000000}\selectfont \(\displaystyle {80}\)}%
\end{pgfscope}%
\begin{pgfscope}%
\pgfsetbuttcap%
\pgfsetroundjoin%
\definecolor{currentfill}{rgb}{0.000000,0.000000,0.000000}%
\pgfsetfillcolor{currentfill}%
\pgfsetlinewidth{0.803000pt}%
\definecolor{currentstroke}{rgb}{0.000000,0.000000,0.000000}%
\pgfsetstrokecolor{currentstroke}%
\pgfsetdash{}{0pt}%
\pgfsys@defobject{currentmarker}{\pgfqpoint{-0.048611in}{0.000000in}}{\pgfqpoint{-0.000000in}{0.000000in}}{%
\pgfpathmoveto{\pgfqpoint{-0.000000in}{0.000000in}}%
\pgfpathlineto{\pgfqpoint{-0.048611in}{0.000000in}}%
\pgfusepath{stroke,fill}%
}%
\begin{pgfscope}%
\pgfsys@transformshift{0.800000in}{4.172016in}%
\pgfsys@useobject{currentmarker}{}%
\end{pgfscope}%
\end{pgfscope}%
\begin{pgfscope}%
\definecolor{textcolor}{rgb}{0.000000,0.000000,0.000000}%
\pgfsetstrokecolor{textcolor}%
\pgfsetfillcolor{textcolor}%
\pgftext[x=0.494444in, y=4.123791in, left, base]{\color{textcolor}\rmfamily\fontsize{10.000000}{12.000000}\selectfont \(\displaystyle {100}\)}%
\end{pgfscope}%
\begin{pgfscope}%
\pgfpathrectangle{\pgfqpoint{0.800000in}{0.528000in}}{\pgfqpoint{4.960000in}{3.696000in}}%
\pgfusepath{clip}%
\pgfsetrectcap%
\pgfsetroundjoin%
\pgfsetlinewidth{1.505625pt}%
\definecolor{currentstroke}{rgb}{0.000000,0.000000,1.000000}%
\pgfsetstrokecolor{currentstroke}%
\pgfsetdash{}{0pt}%
\pgfpathmoveto{\pgfqpoint{1.025455in}{0.696000in}}%
\pgfpathlineto{\pgfqpoint{1.085979in}{0.699743in}}%
\pgfpathlineto{\pgfqpoint{1.146504in}{0.705708in}}%
\pgfpathlineto{\pgfqpoint{1.207029in}{0.713648in}}%
\pgfpathlineto{\pgfqpoint{1.267553in}{0.723412in}}%
\pgfpathlineto{\pgfqpoint{1.328078in}{0.734905in}}%
\pgfpathlineto{\pgfqpoint{1.388603in}{0.748057in}}%
\pgfpathlineto{\pgfqpoint{1.449128in}{0.762816in}}%
\pgfpathlineto{\pgfqpoint{1.539915in}{0.787865in}}%
\pgfpathlineto{\pgfqpoint{1.630702in}{0.816277in}}%
\pgfpathlineto{\pgfqpoint{1.721489in}{0.847900in}}%
\pgfpathlineto{\pgfqpoint{1.812276in}{0.882568in}}%
\pgfpathlineto{\pgfqpoint{1.903063in}{0.920103in}}%
\pgfpathlineto{\pgfqpoint{1.993850in}{0.960313in}}%
\pgfpathlineto{\pgfqpoint{2.084637in}{1.002996in}}%
\pgfpathlineto{\pgfqpoint{2.175424in}{1.047946in}}%
\pgfpathlineto{\pgfqpoint{2.266211in}{1.094959in}}%
\pgfpathlineto{\pgfqpoint{2.387261in}{1.160512in}}%
\pgfpathlineto{\pgfqpoint{2.508310in}{1.228950in}}%
\pgfpathlineto{\pgfqpoint{2.629359in}{1.299907in}}%
\pgfpathlineto{\pgfqpoint{2.750409in}{1.373101in}}%
\pgfpathlineto{\pgfqpoint{2.901721in}{1.467473in}}%
\pgfpathlineto{\pgfqpoint{3.053032in}{1.564950in}}%
\pgfpathlineto{\pgfqpoint{3.174082in}{1.645269in}}%
\pgfpathlineto{\pgfqpoint{3.295131in}{1.727882in}}%
\pgfpathlineto{\pgfqpoint{3.416181in}{1.813088in}}%
\pgfpathlineto{\pgfqpoint{3.537230in}{1.901268in}}%
\pgfpathlineto{\pgfqpoint{3.658279in}{1.992855in}}%
\pgfpathlineto{\pgfqpoint{3.749067in}{2.064061in}}%
\pgfpathlineto{\pgfqpoint{3.839854in}{2.137643in}}%
\pgfpathlineto{\pgfqpoint{3.930641in}{2.213796in}}%
\pgfpathlineto{\pgfqpoint{4.021428in}{2.292699in}}%
\pgfpathlineto{\pgfqpoint{4.112215in}{2.374517in}}%
\pgfpathlineto{\pgfqpoint{4.203002in}{2.459391in}}%
\pgfpathlineto{\pgfqpoint{4.293789in}{2.547432in}}%
\pgfpathlineto{\pgfqpoint{4.384576in}{2.638723in}}%
\pgfpathlineto{\pgfqpoint{4.475363in}{2.733310in}}%
\pgfpathlineto{\pgfqpoint{4.566150in}{2.831205in}}%
\pgfpathlineto{\pgfqpoint{4.656937in}{2.932384in}}%
\pgfpathlineto{\pgfqpoint{4.747724in}{3.036787in}}%
\pgfpathlineto{\pgfqpoint{4.838511in}{3.144321in}}%
\pgfpathlineto{\pgfqpoint{4.929298in}{3.254863in}}%
\pgfpathlineto{\pgfqpoint{5.020085in}{3.368264in}}%
\pgfpathlineto{\pgfqpoint{5.141135in}{3.523614in}}%
\pgfpathlineto{\pgfqpoint{5.262184in}{3.683297in}}%
\pgfpathlineto{\pgfqpoint{5.383234in}{3.846842in}}%
\pgfpathlineto{\pgfqpoint{5.504283in}{4.013785in}}%
\pgfpathlineto{\pgfqpoint{5.534545in}{4.056000in}}%
\pgfpathlineto{\pgfqpoint{5.534545in}{4.056000in}}%
\pgfusepath{stroke}%
\end{pgfscope}%
\begin{pgfscope}%
\pgfpathrectangle{\pgfqpoint{0.800000in}{0.528000in}}{\pgfqpoint{4.960000in}{3.696000in}}%
\pgfusepath{clip}%
\pgfsetrectcap%
\pgfsetroundjoin%
\pgfsetlinewidth{1.505625pt}%
\definecolor{currentstroke}{rgb}{0.000000,0.000000,0.000000}%
\pgfsetstrokecolor{currentstroke}%
\pgfsetdash{}{0pt}%
\pgfpathmoveto{\pgfqpoint{0.800000in}{0.810962in}}%
\pgfpathlineto{\pgfqpoint{5.760000in}{0.810962in}}%
\pgfusepath{stroke}%
\end{pgfscope}%
\begin{pgfscope}%
\pgfsetrectcap%
\pgfsetmiterjoin%
\pgfsetlinewidth{0.803000pt}%
\definecolor{currentstroke}{rgb}{0.000000,0.000000,0.000000}%
\pgfsetstrokecolor{currentstroke}%
\pgfsetdash{}{0pt}%
\pgfpathmoveto{\pgfqpoint{0.800000in}{0.528000in}}%
\pgfpathlineto{\pgfqpoint{0.800000in}{4.224000in}}%
\pgfusepath{stroke}%
\end{pgfscope}%
\begin{pgfscope}%
\pgfsetrectcap%
\pgfsetmiterjoin%
\pgfsetlinewidth{0.803000pt}%
\definecolor{currentstroke}{rgb}{0.000000,0.000000,0.000000}%
\pgfsetstrokecolor{currentstroke}%
\pgfsetdash{}{0pt}%
\pgfpathmoveto{\pgfqpoint{5.760000in}{0.528000in}}%
\pgfpathlineto{\pgfqpoint{5.760000in}{4.224000in}}%
\pgfusepath{stroke}%
\end{pgfscope}%
\begin{pgfscope}%
\pgfsetrectcap%
\pgfsetmiterjoin%
\pgfsetlinewidth{0.803000pt}%
\definecolor{currentstroke}{rgb}{0.000000,0.000000,0.000000}%
\pgfsetstrokecolor{currentstroke}%
\pgfsetdash{}{0pt}%
\pgfpathmoveto{\pgfqpoint{0.800000in}{0.528000in}}%
\pgfpathlineto{\pgfqpoint{5.760000in}{0.528000in}}%
\pgfusepath{stroke}%
\end{pgfscope}%
\begin{pgfscope}%
\pgfsetrectcap%
\pgfsetmiterjoin%
\pgfsetlinewidth{0.803000pt}%
\definecolor{currentstroke}{rgb}{0.000000,0.000000,0.000000}%
\pgfsetstrokecolor{currentstroke}%
\pgfsetdash{}{0pt}%
\pgfpathmoveto{\pgfqpoint{0.800000in}{4.224000in}}%
\pgfpathlineto{\pgfqpoint{5.760000in}{4.224000in}}%
\pgfusepath{stroke}%
\end{pgfscope}%
\begin{pgfscope}%
\definecolor{textcolor}{rgb}{0.000000,0.000000,0.000000}%
\pgfsetstrokecolor{textcolor}%
\pgfsetfillcolor{textcolor}%
\pgftext[x=3.280000in,y=4.307333in,,base]{\color{textcolor}\rmfamily\fontsize{12.000000}{14.400000}\selectfont Riemann integral}%
\end{pgfscope}%
\end{pgfpicture}%
\makeatother%
\endgroup%

	\caption{Całka w sensie Riemanna}
\end{figure}

Wyznaczone pole pomiędzy wykresem funkcji a osią $OX$ wynosi: $\py{trunc(RIEMANN)}$.

\subsection{Całkowanie górne i dolne}
Wizualizacja graficzno-geometryczna całkowania górnego i dolnego. Dla całkowania górnego i dolnego dzielimy nasz przedział $[a, b]$ na $N$ równych części. Całka górna za wysokość prostokąta, przybliżającego fragment pola pod wykresem, przyjmuje największą wartość na aktualnym przedziale $[x_{i-1}, x_i]$. Całka dolna działa podobnie, jednak za wysokość prostokąta przyjmowana jest wartość najmniejsza. Kolor różowy reprezentuje sumy górne, kolor zielony --- sumy dolne, brązowy --- oba słupki się pokrywają.

\begin{figure}[h]
	%% Creator: Matplotlib, PGF backend
%%
%% To include the figure in your LaTeX document, write
%%   \input{<filename>.pgf}
%%
%% Make sure the required packages are loaded in your preamble
%%   \usepackage{pgf}
%%
%% Also ensure that all the required font packages are loaded; for instance,
%% the lmodern package is sometimes necessary when using math font.
%%   \usepackage{lmodern}
%%
%% Figures using additional raster images can only be included by \input if
%% they are in the same directory as the main LaTeX file. For loading figures
%% from other directories you can use the `import` package
%%   \usepackage{import}
%%
%% and then include the figures with
%%   \import{<path to file>}{<filename>.pgf}
%%
%% Matplotlib used the following preamble
%%   
%%   \makeatletter\@ifpackageloaded{underscore}{}{\usepackage[strings]{underscore}}\makeatother
%%
\begingroup%
\makeatletter%
\begin{pgfpicture}%
\pgfpathrectangle{\pgfpointorigin}{\pgfqpoint{6.400000in}{4.800000in}}%
\pgfusepath{use as bounding box, clip}%
\begin{pgfscope}%
\pgfsetbuttcap%
\pgfsetmiterjoin%
\definecolor{currentfill}{rgb}{1.000000,1.000000,1.000000}%
\pgfsetfillcolor{currentfill}%
\pgfsetlinewidth{0.000000pt}%
\definecolor{currentstroke}{rgb}{1.000000,1.000000,1.000000}%
\pgfsetstrokecolor{currentstroke}%
\pgfsetdash{}{0pt}%
\pgfpathmoveto{\pgfqpoint{0.000000in}{0.000000in}}%
\pgfpathlineto{\pgfqpoint{6.400000in}{0.000000in}}%
\pgfpathlineto{\pgfqpoint{6.400000in}{4.800000in}}%
\pgfpathlineto{\pgfqpoint{0.000000in}{4.800000in}}%
\pgfpathlineto{\pgfqpoint{0.000000in}{0.000000in}}%
\pgfpathclose%
\pgfusepath{fill}%
\end{pgfscope}%
\begin{pgfscope}%
\pgfsetbuttcap%
\pgfsetmiterjoin%
\definecolor{currentfill}{rgb}{1.000000,1.000000,1.000000}%
\pgfsetfillcolor{currentfill}%
\pgfsetlinewidth{0.000000pt}%
\definecolor{currentstroke}{rgb}{0.000000,0.000000,0.000000}%
\pgfsetstrokecolor{currentstroke}%
\pgfsetstrokeopacity{0.000000}%
\pgfsetdash{}{0pt}%
\pgfpathmoveto{\pgfqpoint{0.800000in}{0.528000in}}%
\pgfpathlineto{\pgfqpoint{5.760000in}{0.528000in}}%
\pgfpathlineto{\pgfqpoint{5.760000in}{4.224000in}}%
\pgfpathlineto{\pgfqpoint{0.800000in}{4.224000in}}%
\pgfpathlineto{\pgfqpoint{0.800000in}{0.528000in}}%
\pgfpathclose%
\pgfusepath{fill}%
\end{pgfscope}%
\begin{pgfscope}%
\pgfpathrectangle{\pgfqpoint{0.800000in}{0.528000in}}{\pgfqpoint{4.960000in}{3.696000in}}%
\pgfusepath{clip}%
\pgfsetbuttcap%
\pgfsetmiterjoin%
\definecolor{currentfill}{rgb}{1.000000,0.000000,0.000000}%
\pgfsetfillcolor{currentfill}%
\pgfsetfillopacity{0.300000}%
\pgfsetlinewidth{1.003750pt}%
\definecolor{currentstroke}{rgb}{0.000000,0.000000,0.000000}%
\pgfsetstrokecolor{currentstroke}%
\pgfsetstrokeopacity{0.300000}%
\pgfsetdash{}{0pt}%
\pgfpathmoveto{\pgfqpoint{1.326061in}{0.810962in}}%
\pgfpathlineto{\pgfqpoint{1.025455in}{0.810962in}}%
\pgfpathlineto{\pgfqpoint{1.025455in}{0.734495in}}%
\pgfpathlineto{\pgfqpoint{1.326061in}{0.734495in}}%
\pgfpathlineto{\pgfqpoint{1.326061in}{0.810962in}}%
\pgfpathclose%
\pgfusepath{stroke,fill}%
\end{pgfscope}%
\begin{pgfscope}%
\pgfpathrectangle{\pgfqpoint{0.800000in}{0.528000in}}{\pgfqpoint{4.960000in}{3.696000in}}%
\pgfusepath{clip}%
\pgfsetbuttcap%
\pgfsetmiterjoin%
\definecolor{currentfill}{rgb}{1.000000,0.000000,0.000000}%
\pgfsetfillcolor{currentfill}%
\pgfsetfillopacity{0.300000}%
\pgfsetlinewidth{1.003750pt}%
\definecolor{currentstroke}{rgb}{0.000000,0.000000,0.000000}%
\pgfsetstrokecolor{currentstroke}%
\pgfsetstrokeopacity{0.300000}%
\pgfsetdash{}{0pt}%
\pgfpathmoveto{\pgfqpoint{1.626667in}{0.810962in}}%
\pgfpathlineto{\pgfqpoint{1.326061in}{0.810962in}}%
\pgfpathlineto{\pgfqpoint{1.326061in}{0.814945in}}%
\pgfpathlineto{\pgfqpoint{1.626667in}{0.814945in}}%
\pgfpathlineto{\pgfqpoint{1.626667in}{0.810962in}}%
\pgfpathclose%
\pgfusepath{stroke,fill}%
\end{pgfscope}%
\begin{pgfscope}%
\pgfpathrectangle{\pgfqpoint{0.800000in}{0.528000in}}{\pgfqpoint{4.960000in}{3.696000in}}%
\pgfusepath{clip}%
\pgfsetbuttcap%
\pgfsetmiterjoin%
\definecolor{currentfill}{rgb}{1.000000,0.000000,0.000000}%
\pgfsetfillcolor{currentfill}%
\pgfsetfillopacity{0.300000}%
\pgfsetlinewidth{1.003750pt}%
\definecolor{currentstroke}{rgb}{0.000000,0.000000,0.000000}%
\pgfsetstrokecolor{currentstroke}%
\pgfsetstrokeopacity{0.300000}%
\pgfsetdash{}{0pt}%
\pgfpathmoveto{\pgfqpoint{1.927273in}{0.810962in}}%
\pgfpathlineto{\pgfqpoint{1.626667in}{0.810962in}}%
\pgfpathlineto{\pgfqpoint{1.626667in}{0.930573in}}%
\pgfpathlineto{\pgfqpoint{1.927273in}{0.930573in}}%
\pgfpathlineto{\pgfqpoint{1.927273in}{0.810962in}}%
\pgfpathclose%
\pgfusepath{stroke,fill}%
\end{pgfscope}%
\begin{pgfscope}%
\pgfpathrectangle{\pgfqpoint{0.800000in}{0.528000in}}{\pgfqpoint{4.960000in}{3.696000in}}%
\pgfusepath{clip}%
\pgfsetbuttcap%
\pgfsetmiterjoin%
\definecolor{currentfill}{rgb}{1.000000,0.000000,0.000000}%
\pgfsetfillcolor{currentfill}%
\pgfsetfillopacity{0.300000}%
\pgfsetlinewidth{1.003750pt}%
\definecolor{currentstroke}{rgb}{0.000000,0.000000,0.000000}%
\pgfsetstrokecolor{currentstroke}%
\pgfsetstrokeopacity{0.300000}%
\pgfsetdash{}{0pt}%
\pgfpathmoveto{\pgfqpoint{2.227879in}{0.810962in}}%
\pgfpathlineto{\pgfqpoint{1.927273in}{0.810962in}}%
\pgfpathlineto{\pgfqpoint{1.927273in}{1.074871in}}%
\pgfpathlineto{\pgfqpoint{2.227879in}{1.074871in}}%
\pgfpathlineto{\pgfqpoint{2.227879in}{0.810962in}}%
\pgfpathclose%
\pgfusepath{stroke,fill}%
\end{pgfscope}%
\begin{pgfscope}%
\pgfpathrectangle{\pgfqpoint{0.800000in}{0.528000in}}{\pgfqpoint{4.960000in}{3.696000in}}%
\pgfusepath{clip}%
\pgfsetbuttcap%
\pgfsetmiterjoin%
\definecolor{currentfill}{rgb}{1.000000,0.000000,0.000000}%
\pgfsetfillcolor{currentfill}%
\pgfsetfillopacity{0.300000}%
\pgfsetlinewidth{1.003750pt}%
\definecolor{currentstroke}{rgb}{0.000000,0.000000,0.000000}%
\pgfsetstrokecolor{currentstroke}%
\pgfsetstrokeopacity{0.300000}%
\pgfsetdash{}{0pt}%
\pgfpathmoveto{\pgfqpoint{2.528485in}{0.810962in}}%
\pgfpathlineto{\pgfqpoint{2.227879in}{0.810962in}}%
\pgfpathlineto{\pgfqpoint{2.227879in}{1.240610in}}%
\pgfpathlineto{\pgfqpoint{2.528485in}{1.240610in}}%
\pgfpathlineto{\pgfqpoint{2.528485in}{0.810962in}}%
\pgfpathclose%
\pgfusepath{stroke,fill}%
\end{pgfscope}%
\begin{pgfscope}%
\pgfpathrectangle{\pgfqpoint{0.800000in}{0.528000in}}{\pgfqpoint{4.960000in}{3.696000in}}%
\pgfusepath{clip}%
\pgfsetbuttcap%
\pgfsetmiterjoin%
\definecolor{currentfill}{rgb}{1.000000,0.000000,0.000000}%
\pgfsetfillcolor{currentfill}%
\pgfsetfillopacity{0.300000}%
\pgfsetlinewidth{1.003750pt}%
\definecolor{currentstroke}{rgb}{0.000000,0.000000,0.000000}%
\pgfsetstrokecolor{currentstroke}%
\pgfsetstrokeopacity{0.300000}%
\pgfsetdash{}{0pt}%
\pgfpathmoveto{\pgfqpoint{2.829091in}{0.810962in}}%
\pgfpathlineto{\pgfqpoint{2.528485in}{0.810962in}}%
\pgfpathlineto{\pgfqpoint{2.528485in}{1.421786in}}%
\pgfpathlineto{\pgfqpoint{2.829091in}{1.421786in}}%
\pgfpathlineto{\pgfqpoint{2.829091in}{0.810962in}}%
\pgfpathclose%
\pgfusepath{stroke,fill}%
\end{pgfscope}%
\begin{pgfscope}%
\pgfpathrectangle{\pgfqpoint{0.800000in}{0.528000in}}{\pgfqpoint{4.960000in}{3.696000in}}%
\pgfusepath{clip}%
\pgfsetbuttcap%
\pgfsetmiterjoin%
\definecolor{currentfill}{rgb}{1.000000,0.000000,0.000000}%
\pgfsetfillcolor{currentfill}%
\pgfsetfillopacity{0.300000}%
\pgfsetlinewidth{1.003750pt}%
\definecolor{currentstroke}{rgb}{0.000000,0.000000,0.000000}%
\pgfsetstrokecolor{currentstroke}%
\pgfsetstrokeopacity{0.300000}%
\pgfsetdash{}{0pt}%
\pgfpathmoveto{\pgfqpoint{3.129697in}{0.810962in}}%
\pgfpathlineto{\pgfqpoint{2.829091in}{0.810962in}}%
\pgfpathlineto{\pgfqpoint{2.829091in}{1.615566in}}%
\pgfpathlineto{\pgfqpoint{3.129697in}{1.615566in}}%
\pgfpathlineto{\pgfqpoint{3.129697in}{0.810962in}}%
\pgfpathclose%
\pgfusepath{stroke,fill}%
\end{pgfscope}%
\begin{pgfscope}%
\pgfpathrectangle{\pgfqpoint{0.800000in}{0.528000in}}{\pgfqpoint{4.960000in}{3.696000in}}%
\pgfusepath{clip}%
\pgfsetbuttcap%
\pgfsetmiterjoin%
\definecolor{currentfill}{rgb}{1.000000,0.000000,0.000000}%
\pgfsetfillcolor{currentfill}%
\pgfsetfillopacity{0.300000}%
\pgfsetlinewidth{1.003750pt}%
\definecolor{currentstroke}{rgb}{0.000000,0.000000,0.000000}%
\pgfsetstrokecolor{currentstroke}%
\pgfsetstrokeopacity{0.300000}%
\pgfsetdash{}{0pt}%
\pgfpathmoveto{\pgfqpoint{3.430303in}{0.810962in}}%
\pgfpathlineto{\pgfqpoint{3.129697in}{0.810962in}}%
\pgfpathlineto{\pgfqpoint{3.129697in}{1.823215in}}%
\pgfpathlineto{\pgfqpoint{3.430303in}{1.823215in}}%
\pgfpathlineto{\pgfqpoint{3.430303in}{0.810962in}}%
\pgfpathclose%
\pgfusepath{stroke,fill}%
\end{pgfscope}%
\begin{pgfscope}%
\pgfpathrectangle{\pgfqpoint{0.800000in}{0.528000in}}{\pgfqpoint{4.960000in}{3.696000in}}%
\pgfusepath{clip}%
\pgfsetbuttcap%
\pgfsetmiterjoin%
\definecolor{currentfill}{rgb}{1.000000,0.000000,0.000000}%
\pgfsetfillcolor{currentfill}%
\pgfsetfillopacity{0.300000}%
\pgfsetlinewidth{1.003750pt}%
\definecolor{currentstroke}{rgb}{0.000000,0.000000,0.000000}%
\pgfsetstrokecolor{currentstroke}%
\pgfsetstrokeopacity{0.300000}%
\pgfsetdash{}{0pt}%
\pgfpathmoveto{\pgfqpoint{3.730909in}{0.810962in}}%
\pgfpathlineto{\pgfqpoint{3.430303in}{0.810962in}}%
\pgfpathlineto{\pgfqpoint{3.430303in}{2.049636in}}%
\pgfpathlineto{\pgfqpoint{3.730909in}{2.049636in}}%
\pgfpathlineto{\pgfqpoint{3.730909in}{0.810962in}}%
\pgfpathclose%
\pgfusepath{stroke,fill}%
\end{pgfscope}%
\begin{pgfscope}%
\pgfpathrectangle{\pgfqpoint{0.800000in}{0.528000in}}{\pgfqpoint{4.960000in}{3.696000in}}%
\pgfusepath{clip}%
\pgfsetbuttcap%
\pgfsetmiterjoin%
\definecolor{currentfill}{rgb}{1.000000,0.000000,0.000000}%
\pgfsetfillcolor{currentfill}%
\pgfsetfillopacity{0.300000}%
\pgfsetlinewidth{1.003750pt}%
\definecolor{currentstroke}{rgb}{0.000000,0.000000,0.000000}%
\pgfsetstrokecolor{currentstroke}%
\pgfsetstrokeopacity{0.300000}%
\pgfsetdash{}{0pt}%
\pgfpathmoveto{\pgfqpoint{4.031515in}{0.810962in}}%
\pgfpathlineto{\pgfqpoint{3.730909in}{0.810962in}}%
\pgfpathlineto{\pgfqpoint{3.730909in}{2.301643in}}%
\pgfpathlineto{\pgfqpoint{4.031515in}{2.301643in}}%
\pgfpathlineto{\pgfqpoint{4.031515in}{0.810962in}}%
\pgfpathclose%
\pgfusepath{stroke,fill}%
\end{pgfscope}%
\begin{pgfscope}%
\pgfpathrectangle{\pgfqpoint{0.800000in}{0.528000in}}{\pgfqpoint{4.960000in}{3.696000in}}%
\pgfusepath{clip}%
\pgfsetbuttcap%
\pgfsetmiterjoin%
\definecolor{currentfill}{rgb}{1.000000,0.000000,0.000000}%
\pgfsetfillcolor{currentfill}%
\pgfsetfillopacity{0.300000}%
\pgfsetlinewidth{1.003750pt}%
\definecolor{currentstroke}{rgb}{0.000000,0.000000,0.000000}%
\pgfsetstrokecolor{currentstroke}%
\pgfsetstrokeopacity{0.300000}%
\pgfsetdash{}{0pt}%
\pgfpathmoveto{\pgfqpoint{4.332121in}{0.810962in}}%
\pgfpathlineto{\pgfqpoint{4.031515in}{0.810962in}}%
\pgfpathlineto{\pgfqpoint{4.031515in}{2.585577in}}%
\pgfpathlineto{\pgfqpoint{4.332121in}{2.585577in}}%
\pgfpathlineto{\pgfqpoint{4.332121in}{0.810962in}}%
\pgfpathclose%
\pgfusepath{stroke,fill}%
\end{pgfscope}%
\begin{pgfscope}%
\pgfpathrectangle{\pgfqpoint{0.800000in}{0.528000in}}{\pgfqpoint{4.960000in}{3.696000in}}%
\pgfusepath{clip}%
\pgfsetbuttcap%
\pgfsetmiterjoin%
\definecolor{currentfill}{rgb}{1.000000,0.000000,0.000000}%
\pgfsetfillcolor{currentfill}%
\pgfsetfillopacity{0.300000}%
\pgfsetlinewidth{1.003750pt}%
\definecolor{currentstroke}{rgb}{0.000000,0.000000,0.000000}%
\pgfsetstrokecolor{currentstroke}%
\pgfsetstrokeopacity{0.300000}%
\pgfsetdash{}{0pt}%
\pgfpathmoveto{\pgfqpoint{4.632727in}{0.810962in}}%
\pgfpathlineto{\pgfqpoint{4.332121in}{0.810962in}}%
\pgfpathlineto{\pgfqpoint{4.332121in}{2.905084in}}%
\pgfpathlineto{\pgfqpoint{4.632727in}{2.905084in}}%
\pgfpathlineto{\pgfqpoint{4.632727in}{0.810962in}}%
\pgfpathclose%
\pgfusepath{stroke,fill}%
\end{pgfscope}%
\begin{pgfscope}%
\pgfpathrectangle{\pgfqpoint{0.800000in}{0.528000in}}{\pgfqpoint{4.960000in}{3.696000in}}%
\pgfusepath{clip}%
\pgfsetbuttcap%
\pgfsetmiterjoin%
\definecolor{currentfill}{rgb}{1.000000,0.000000,0.000000}%
\pgfsetfillcolor{currentfill}%
\pgfsetfillopacity{0.300000}%
\pgfsetlinewidth{1.003750pt}%
\definecolor{currentstroke}{rgb}{0.000000,0.000000,0.000000}%
\pgfsetstrokecolor{currentstroke}%
\pgfsetstrokeopacity{0.300000}%
\pgfsetdash{}{0pt}%
\pgfpathmoveto{\pgfqpoint{4.933333in}{0.810962in}}%
\pgfpathlineto{\pgfqpoint{4.632727in}{0.810962in}}%
\pgfpathlineto{\pgfqpoint{4.632727in}{3.259843in}}%
\pgfpathlineto{\pgfqpoint{4.933333in}{3.259843in}}%
\pgfpathlineto{\pgfqpoint{4.933333in}{0.810962in}}%
\pgfpathclose%
\pgfusepath{stroke,fill}%
\end{pgfscope}%
\begin{pgfscope}%
\pgfpathrectangle{\pgfqpoint{0.800000in}{0.528000in}}{\pgfqpoint{4.960000in}{3.696000in}}%
\pgfusepath{clip}%
\pgfsetbuttcap%
\pgfsetmiterjoin%
\definecolor{currentfill}{rgb}{1.000000,0.000000,0.000000}%
\pgfsetfillcolor{currentfill}%
\pgfsetfillopacity{0.300000}%
\pgfsetlinewidth{1.003750pt}%
\definecolor{currentstroke}{rgb}{0.000000,0.000000,0.000000}%
\pgfsetstrokecolor{currentstroke}%
\pgfsetstrokeopacity{0.300000}%
\pgfsetdash{}{0pt}%
\pgfpathmoveto{\pgfqpoint{5.233939in}{0.810962in}}%
\pgfpathlineto{\pgfqpoint{4.933333in}{0.810962in}}%
\pgfpathlineto{\pgfqpoint{4.933333in}{3.645674in}}%
\pgfpathlineto{\pgfqpoint{5.233939in}{3.645674in}}%
\pgfpathlineto{\pgfqpoint{5.233939in}{0.810962in}}%
\pgfpathclose%
\pgfusepath{stroke,fill}%
\end{pgfscope}%
\begin{pgfscope}%
\pgfpathrectangle{\pgfqpoint{0.800000in}{0.528000in}}{\pgfqpoint{4.960000in}{3.696000in}}%
\pgfusepath{clip}%
\pgfsetbuttcap%
\pgfsetmiterjoin%
\definecolor{currentfill}{rgb}{1.000000,0.000000,0.000000}%
\pgfsetfillcolor{currentfill}%
\pgfsetfillopacity{0.300000}%
\pgfsetlinewidth{1.003750pt}%
\definecolor{currentstroke}{rgb}{0.000000,0.000000,0.000000}%
\pgfsetstrokecolor{currentstroke}%
\pgfsetstrokeopacity{0.300000}%
\pgfsetdash{}{0pt}%
\pgfpathmoveto{\pgfqpoint{5.534545in}{0.810962in}}%
\pgfpathlineto{\pgfqpoint{5.233939in}{0.810962in}}%
\pgfpathlineto{\pgfqpoint{5.233939in}{4.056000in}}%
\pgfpathlineto{\pgfqpoint{5.534545in}{4.056000in}}%
\pgfpathlineto{\pgfqpoint{5.534545in}{0.810962in}}%
\pgfpathclose%
\pgfusepath{stroke,fill}%
\end{pgfscope}%
\begin{pgfscope}%
\pgfpathrectangle{\pgfqpoint{0.800000in}{0.528000in}}{\pgfqpoint{4.960000in}{3.696000in}}%
\pgfusepath{clip}%
\pgfsetbuttcap%
\pgfsetmiterjoin%
\definecolor{currentfill}{rgb}{0.000000,0.501961,0.000000}%
\pgfsetfillcolor{currentfill}%
\pgfsetfillopacity{0.300000}%
\pgfsetlinewidth{1.003750pt}%
\definecolor{currentstroke}{rgb}{0.000000,0.000000,0.000000}%
\pgfsetstrokecolor{currentstroke}%
\pgfsetstrokeopacity{0.300000}%
\pgfsetdash{}{0pt}%
\pgfpathmoveto{\pgfqpoint{1.326061in}{0.810962in}}%
\pgfpathlineto{\pgfqpoint{1.025455in}{0.810962in}}%
\pgfpathlineto{\pgfqpoint{1.025455in}{0.696000in}}%
\pgfpathlineto{\pgfqpoint{1.326061in}{0.696000in}}%
\pgfpathlineto{\pgfqpoint{1.326061in}{0.810962in}}%
\pgfpathclose%
\pgfusepath{stroke,fill}%
\end{pgfscope}%
\begin{pgfscope}%
\pgfpathrectangle{\pgfqpoint{0.800000in}{0.528000in}}{\pgfqpoint{4.960000in}{3.696000in}}%
\pgfusepath{clip}%
\pgfsetbuttcap%
\pgfsetmiterjoin%
\definecolor{currentfill}{rgb}{0.000000,0.501961,0.000000}%
\pgfsetfillcolor{currentfill}%
\pgfsetfillopacity{0.300000}%
\pgfsetlinewidth{1.003750pt}%
\definecolor{currentstroke}{rgb}{0.000000,0.000000,0.000000}%
\pgfsetstrokecolor{currentstroke}%
\pgfsetstrokeopacity{0.300000}%
\pgfsetdash{}{0pt}%
\pgfpathmoveto{\pgfqpoint{1.626667in}{0.810962in}}%
\pgfpathlineto{\pgfqpoint{1.326061in}{0.810962in}}%
\pgfpathlineto{\pgfqpoint{1.326061in}{0.734495in}}%
\pgfpathlineto{\pgfqpoint{1.626667in}{0.734495in}}%
\pgfpathlineto{\pgfqpoint{1.626667in}{0.810962in}}%
\pgfpathclose%
\pgfusepath{stroke,fill}%
\end{pgfscope}%
\begin{pgfscope}%
\pgfpathrectangle{\pgfqpoint{0.800000in}{0.528000in}}{\pgfqpoint{4.960000in}{3.696000in}}%
\pgfusepath{clip}%
\pgfsetbuttcap%
\pgfsetmiterjoin%
\definecolor{currentfill}{rgb}{0.000000,0.501961,0.000000}%
\pgfsetfillcolor{currentfill}%
\pgfsetfillopacity{0.300000}%
\pgfsetlinewidth{1.003750pt}%
\definecolor{currentstroke}{rgb}{0.000000,0.000000,0.000000}%
\pgfsetstrokecolor{currentstroke}%
\pgfsetstrokeopacity{0.300000}%
\pgfsetdash{}{0pt}%
\pgfpathmoveto{\pgfqpoint{1.927273in}{0.810962in}}%
\pgfpathlineto{\pgfqpoint{1.626667in}{0.810962in}}%
\pgfpathlineto{\pgfqpoint{1.626667in}{0.814945in}}%
\pgfpathlineto{\pgfqpoint{1.927273in}{0.814945in}}%
\pgfpathlineto{\pgfqpoint{1.927273in}{0.810962in}}%
\pgfpathclose%
\pgfusepath{stroke,fill}%
\end{pgfscope}%
\begin{pgfscope}%
\pgfpathrectangle{\pgfqpoint{0.800000in}{0.528000in}}{\pgfqpoint{4.960000in}{3.696000in}}%
\pgfusepath{clip}%
\pgfsetbuttcap%
\pgfsetmiterjoin%
\definecolor{currentfill}{rgb}{0.000000,0.501961,0.000000}%
\pgfsetfillcolor{currentfill}%
\pgfsetfillopacity{0.300000}%
\pgfsetlinewidth{1.003750pt}%
\definecolor{currentstroke}{rgb}{0.000000,0.000000,0.000000}%
\pgfsetstrokecolor{currentstroke}%
\pgfsetstrokeopacity{0.300000}%
\pgfsetdash{}{0pt}%
\pgfpathmoveto{\pgfqpoint{2.227879in}{0.810962in}}%
\pgfpathlineto{\pgfqpoint{1.927273in}{0.810962in}}%
\pgfpathlineto{\pgfqpoint{1.927273in}{0.930573in}}%
\pgfpathlineto{\pgfqpoint{2.227879in}{0.930573in}}%
\pgfpathlineto{\pgfqpoint{2.227879in}{0.810962in}}%
\pgfpathclose%
\pgfusepath{stroke,fill}%
\end{pgfscope}%
\begin{pgfscope}%
\pgfpathrectangle{\pgfqpoint{0.800000in}{0.528000in}}{\pgfqpoint{4.960000in}{3.696000in}}%
\pgfusepath{clip}%
\pgfsetbuttcap%
\pgfsetmiterjoin%
\definecolor{currentfill}{rgb}{0.000000,0.501961,0.000000}%
\pgfsetfillcolor{currentfill}%
\pgfsetfillopacity{0.300000}%
\pgfsetlinewidth{1.003750pt}%
\definecolor{currentstroke}{rgb}{0.000000,0.000000,0.000000}%
\pgfsetstrokecolor{currentstroke}%
\pgfsetstrokeopacity{0.300000}%
\pgfsetdash{}{0pt}%
\pgfpathmoveto{\pgfqpoint{2.528485in}{0.810962in}}%
\pgfpathlineto{\pgfqpoint{2.227879in}{0.810962in}}%
\pgfpathlineto{\pgfqpoint{2.227879in}{1.074871in}}%
\pgfpathlineto{\pgfqpoint{2.528485in}{1.074871in}}%
\pgfpathlineto{\pgfqpoint{2.528485in}{0.810962in}}%
\pgfpathclose%
\pgfusepath{stroke,fill}%
\end{pgfscope}%
\begin{pgfscope}%
\pgfpathrectangle{\pgfqpoint{0.800000in}{0.528000in}}{\pgfqpoint{4.960000in}{3.696000in}}%
\pgfusepath{clip}%
\pgfsetbuttcap%
\pgfsetmiterjoin%
\definecolor{currentfill}{rgb}{0.000000,0.501961,0.000000}%
\pgfsetfillcolor{currentfill}%
\pgfsetfillopacity{0.300000}%
\pgfsetlinewidth{1.003750pt}%
\definecolor{currentstroke}{rgb}{0.000000,0.000000,0.000000}%
\pgfsetstrokecolor{currentstroke}%
\pgfsetstrokeopacity{0.300000}%
\pgfsetdash{}{0pt}%
\pgfpathmoveto{\pgfqpoint{2.829091in}{0.810962in}}%
\pgfpathlineto{\pgfqpoint{2.528485in}{0.810962in}}%
\pgfpathlineto{\pgfqpoint{2.528485in}{1.240610in}}%
\pgfpathlineto{\pgfqpoint{2.829091in}{1.240610in}}%
\pgfpathlineto{\pgfqpoint{2.829091in}{0.810962in}}%
\pgfpathclose%
\pgfusepath{stroke,fill}%
\end{pgfscope}%
\begin{pgfscope}%
\pgfpathrectangle{\pgfqpoint{0.800000in}{0.528000in}}{\pgfqpoint{4.960000in}{3.696000in}}%
\pgfusepath{clip}%
\pgfsetbuttcap%
\pgfsetmiterjoin%
\definecolor{currentfill}{rgb}{0.000000,0.501961,0.000000}%
\pgfsetfillcolor{currentfill}%
\pgfsetfillopacity{0.300000}%
\pgfsetlinewidth{1.003750pt}%
\definecolor{currentstroke}{rgb}{0.000000,0.000000,0.000000}%
\pgfsetstrokecolor{currentstroke}%
\pgfsetstrokeopacity{0.300000}%
\pgfsetdash{}{0pt}%
\pgfpathmoveto{\pgfqpoint{3.129697in}{0.810962in}}%
\pgfpathlineto{\pgfqpoint{2.829091in}{0.810962in}}%
\pgfpathlineto{\pgfqpoint{2.829091in}{1.421786in}}%
\pgfpathlineto{\pgfqpoint{3.129697in}{1.421786in}}%
\pgfpathlineto{\pgfqpoint{3.129697in}{0.810962in}}%
\pgfpathclose%
\pgfusepath{stroke,fill}%
\end{pgfscope}%
\begin{pgfscope}%
\pgfpathrectangle{\pgfqpoint{0.800000in}{0.528000in}}{\pgfqpoint{4.960000in}{3.696000in}}%
\pgfusepath{clip}%
\pgfsetbuttcap%
\pgfsetmiterjoin%
\definecolor{currentfill}{rgb}{0.000000,0.501961,0.000000}%
\pgfsetfillcolor{currentfill}%
\pgfsetfillopacity{0.300000}%
\pgfsetlinewidth{1.003750pt}%
\definecolor{currentstroke}{rgb}{0.000000,0.000000,0.000000}%
\pgfsetstrokecolor{currentstroke}%
\pgfsetstrokeopacity{0.300000}%
\pgfsetdash{}{0pt}%
\pgfpathmoveto{\pgfqpoint{3.430303in}{0.810962in}}%
\pgfpathlineto{\pgfqpoint{3.129697in}{0.810962in}}%
\pgfpathlineto{\pgfqpoint{3.129697in}{1.615566in}}%
\pgfpathlineto{\pgfqpoint{3.430303in}{1.615566in}}%
\pgfpathlineto{\pgfqpoint{3.430303in}{0.810962in}}%
\pgfpathclose%
\pgfusepath{stroke,fill}%
\end{pgfscope}%
\begin{pgfscope}%
\pgfpathrectangle{\pgfqpoint{0.800000in}{0.528000in}}{\pgfqpoint{4.960000in}{3.696000in}}%
\pgfusepath{clip}%
\pgfsetbuttcap%
\pgfsetmiterjoin%
\definecolor{currentfill}{rgb}{0.000000,0.501961,0.000000}%
\pgfsetfillcolor{currentfill}%
\pgfsetfillopacity{0.300000}%
\pgfsetlinewidth{1.003750pt}%
\definecolor{currentstroke}{rgb}{0.000000,0.000000,0.000000}%
\pgfsetstrokecolor{currentstroke}%
\pgfsetstrokeopacity{0.300000}%
\pgfsetdash{}{0pt}%
\pgfpathmoveto{\pgfqpoint{3.730909in}{0.810962in}}%
\pgfpathlineto{\pgfqpoint{3.430303in}{0.810962in}}%
\pgfpathlineto{\pgfqpoint{3.430303in}{1.823215in}}%
\pgfpathlineto{\pgfqpoint{3.730909in}{1.823215in}}%
\pgfpathlineto{\pgfqpoint{3.730909in}{0.810962in}}%
\pgfpathclose%
\pgfusepath{stroke,fill}%
\end{pgfscope}%
\begin{pgfscope}%
\pgfpathrectangle{\pgfqpoint{0.800000in}{0.528000in}}{\pgfqpoint{4.960000in}{3.696000in}}%
\pgfusepath{clip}%
\pgfsetbuttcap%
\pgfsetmiterjoin%
\definecolor{currentfill}{rgb}{0.000000,0.501961,0.000000}%
\pgfsetfillcolor{currentfill}%
\pgfsetfillopacity{0.300000}%
\pgfsetlinewidth{1.003750pt}%
\definecolor{currentstroke}{rgb}{0.000000,0.000000,0.000000}%
\pgfsetstrokecolor{currentstroke}%
\pgfsetstrokeopacity{0.300000}%
\pgfsetdash{}{0pt}%
\pgfpathmoveto{\pgfqpoint{4.031515in}{0.810962in}}%
\pgfpathlineto{\pgfqpoint{3.730909in}{0.810962in}}%
\pgfpathlineto{\pgfqpoint{3.730909in}{2.049636in}}%
\pgfpathlineto{\pgfqpoint{4.031515in}{2.049636in}}%
\pgfpathlineto{\pgfqpoint{4.031515in}{0.810962in}}%
\pgfpathclose%
\pgfusepath{stroke,fill}%
\end{pgfscope}%
\begin{pgfscope}%
\pgfpathrectangle{\pgfqpoint{0.800000in}{0.528000in}}{\pgfqpoint{4.960000in}{3.696000in}}%
\pgfusepath{clip}%
\pgfsetbuttcap%
\pgfsetmiterjoin%
\definecolor{currentfill}{rgb}{0.000000,0.501961,0.000000}%
\pgfsetfillcolor{currentfill}%
\pgfsetfillopacity{0.300000}%
\pgfsetlinewidth{1.003750pt}%
\definecolor{currentstroke}{rgb}{0.000000,0.000000,0.000000}%
\pgfsetstrokecolor{currentstroke}%
\pgfsetstrokeopacity{0.300000}%
\pgfsetdash{}{0pt}%
\pgfpathmoveto{\pgfqpoint{4.332121in}{0.810962in}}%
\pgfpathlineto{\pgfqpoint{4.031515in}{0.810962in}}%
\pgfpathlineto{\pgfqpoint{4.031515in}{2.301643in}}%
\pgfpathlineto{\pgfqpoint{4.332121in}{2.301643in}}%
\pgfpathlineto{\pgfqpoint{4.332121in}{0.810962in}}%
\pgfpathclose%
\pgfusepath{stroke,fill}%
\end{pgfscope}%
\begin{pgfscope}%
\pgfpathrectangle{\pgfqpoint{0.800000in}{0.528000in}}{\pgfqpoint{4.960000in}{3.696000in}}%
\pgfusepath{clip}%
\pgfsetbuttcap%
\pgfsetmiterjoin%
\definecolor{currentfill}{rgb}{0.000000,0.501961,0.000000}%
\pgfsetfillcolor{currentfill}%
\pgfsetfillopacity{0.300000}%
\pgfsetlinewidth{1.003750pt}%
\definecolor{currentstroke}{rgb}{0.000000,0.000000,0.000000}%
\pgfsetstrokecolor{currentstroke}%
\pgfsetstrokeopacity{0.300000}%
\pgfsetdash{}{0pt}%
\pgfpathmoveto{\pgfqpoint{4.632727in}{0.810962in}}%
\pgfpathlineto{\pgfqpoint{4.332121in}{0.810962in}}%
\pgfpathlineto{\pgfqpoint{4.332121in}{2.585577in}}%
\pgfpathlineto{\pgfqpoint{4.632727in}{2.585577in}}%
\pgfpathlineto{\pgfqpoint{4.632727in}{0.810962in}}%
\pgfpathclose%
\pgfusepath{stroke,fill}%
\end{pgfscope}%
\begin{pgfscope}%
\pgfpathrectangle{\pgfqpoint{0.800000in}{0.528000in}}{\pgfqpoint{4.960000in}{3.696000in}}%
\pgfusepath{clip}%
\pgfsetbuttcap%
\pgfsetmiterjoin%
\definecolor{currentfill}{rgb}{0.000000,0.501961,0.000000}%
\pgfsetfillcolor{currentfill}%
\pgfsetfillopacity{0.300000}%
\pgfsetlinewidth{1.003750pt}%
\definecolor{currentstroke}{rgb}{0.000000,0.000000,0.000000}%
\pgfsetstrokecolor{currentstroke}%
\pgfsetstrokeopacity{0.300000}%
\pgfsetdash{}{0pt}%
\pgfpathmoveto{\pgfqpoint{4.933333in}{0.810962in}}%
\pgfpathlineto{\pgfqpoint{4.632727in}{0.810962in}}%
\pgfpathlineto{\pgfqpoint{4.632727in}{2.905084in}}%
\pgfpathlineto{\pgfqpoint{4.933333in}{2.905084in}}%
\pgfpathlineto{\pgfqpoint{4.933333in}{0.810962in}}%
\pgfpathclose%
\pgfusepath{stroke,fill}%
\end{pgfscope}%
\begin{pgfscope}%
\pgfpathrectangle{\pgfqpoint{0.800000in}{0.528000in}}{\pgfqpoint{4.960000in}{3.696000in}}%
\pgfusepath{clip}%
\pgfsetbuttcap%
\pgfsetmiterjoin%
\definecolor{currentfill}{rgb}{0.000000,0.501961,0.000000}%
\pgfsetfillcolor{currentfill}%
\pgfsetfillopacity{0.300000}%
\pgfsetlinewidth{1.003750pt}%
\definecolor{currentstroke}{rgb}{0.000000,0.000000,0.000000}%
\pgfsetstrokecolor{currentstroke}%
\pgfsetstrokeopacity{0.300000}%
\pgfsetdash{}{0pt}%
\pgfpathmoveto{\pgfqpoint{5.233939in}{0.810962in}}%
\pgfpathlineto{\pgfqpoint{4.933333in}{0.810962in}}%
\pgfpathlineto{\pgfqpoint{4.933333in}{3.259843in}}%
\pgfpathlineto{\pgfqpoint{5.233939in}{3.259843in}}%
\pgfpathlineto{\pgfqpoint{5.233939in}{0.810962in}}%
\pgfpathclose%
\pgfusepath{stroke,fill}%
\end{pgfscope}%
\begin{pgfscope}%
\pgfpathrectangle{\pgfqpoint{0.800000in}{0.528000in}}{\pgfqpoint{4.960000in}{3.696000in}}%
\pgfusepath{clip}%
\pgfsetbuttcap%
\pgfsetmiterjoin%
\definecolor{currentfill}{rgb}{0.000000,0.501961,0.000000}%
\pgfsetfillcolor{currentfill}%
\pgfsetfillopacity{0.300000}%
\pgfsetlinewidth{1.003750pt}%
\definecolor{currentstroke}{rgb}{0.000000,0.000000,0.000000}%
\pgfsetstrokecolor{currentstroke}%
\pgfsetstrokeopacity{0.300000}%
\pgfsetdash{}{0pt}%
\pgfpathmoveto{\pgfqpoint{5.534545in}{0.810962in}}%
\pgfpathlineto{\pgfqpoint{5.233939in}{0.810962in}}%
\pgfpathlineto{\pgfqpoint{5.233939in}{3.645674in}}%
\pgfpathlineto{\pgfqpoint{5.534545in}{3.645674in}}%
\pgfpathlineto{\pgfqpoint{5.534545in}{0.810962in}}%
\pgfpathclose%
\pgfusepath{stroke,fill}%
\end{pgfscope}%
\begin{pgfscope}%
\pgfsetbuttcap%
\pgfsetroundjoin%
\definecolor{currentfill}{rgb}{0.000000,0.000000,0.000000}%
\pgfsetfillcolor{currentfill}%
\pgfsetlinewidth{0.803000pt}%
\definecolor{currentstroke}{rgb}{0.000000,0.000000,0.000000}%
\pgfsetstrokecolor{currentstroke}%
\pgfsetdash{}{0pt}%
\pgfsys@defobject{currentmarker}{\pgfqpoint{0.000000in}{-0.048611in}}{\pgfqpoint{0.000000in}{0.000000in}}{%
\pgfpathmoveto{\pgfqpoint{0.000000in}{0.000000in}}%
\pgfpathlineto{\pgfqpoint{0.000000in}{-0.048611in}}%
\pgfusepath{stroke,fill}%
}%
\begin{pgfscope}%
\pgfsys@transformshift{1.526465in}{0.528000in}%
\pgfsys@useobject{currentmarker}{}%
\end{pgfscope}%
\end{pgfscope}%
\begin{pgfscope}%
\definecolor{textcolor}{rgb}{0.000000,0.000000,0.000000}%
\pgfsetstrokecolor{textcolor}%
\pgfsetfillcolor{textcolor}%
\pgftext[x=1.526465in,y=0.430778in,,top]{\color{textcolor}\rmfamily\fontsize{10.000000}{12.000000}\selectfont \(\displaystyle {2}\)}%
\end{pgfscope}%
\begin{pgfscope}%
\pgfsetbuttcap%
\pgfsetroundjoin%
\definecolor{currentfill}{rgb}{0.000000,0.000000,0.000000}%
\pgfsetfillcolor{currentfill}%
\pgfsetlinewidth{0.803000pt}%
\definecolor{currentstroke}{rgb}{0.000000,0.000000,0.000000}%
\pgfsetstrokecolor{currentstroke}%
\pgfsetdash{}{0pt}%
\pgfsys@defobject{currentmarker}{\pgfqpoint{0.000000in}{-0.048611in}}{\pgfqpoint{0.000000in}{0.000000in}}{%
\pgfpathmoveto{\pgfqpoint{0.000000in}{0.000000in}}%
\pgfpathlineto{\pgfqpoint{0.000000in}{-0.048611in}}%
\pgfusepath{stroke,fill}%
}%
\begin{pgfscope}%
\pgfsys@transformshift{2.528485in}{0.528000in}%
\pgfsys@useobject{currentmarker}{}%
\end{pgfscope}%
\end{pgfscope}%
\begin{pgfscope}%
\definecolor{textcolor}{rgb}{0.000000,0.000000,0.000000}%
\pgfsetstrokecolor{textcolor}%
\pgfsetfillcolor{textcolor}%
\pgftext[x=2.528485in,y=0.430778in,,top]{\color{textcolor}\rmfamily\fontsize{10.000000}{12.000000}\selectfont \(\displaystyle {4}\)}%
\end{pgfscope}%
\begin{pgfscope}%
\pgfsetbuttcap%
\pgfsetroundjoin%
\definecolor{currentfill}{rgb}{0.000000,0.000000,0.000000}%
\pgfsetfillcolor{currentfill}%
\pgfsetlinewidth{0.803000pt}%
\definecolor{currentstroke}{rgb}{0.000000,0.000000,0.000000}%
\pgfsetstrokecolor{currentstroke}%
\pgfsetdash{}{0pt}%
\pgfsys@defobject{currentmarker}{\pgfqpoint{0.000000in}{-0.048611in}}{\pgfqpoint{0.000000in}{0.000000in}}{%
\pgfpathmoveto{\pgfqpoint{0.000000in}{0.000000in}}%
\pgfpathlineto{\pgfqpoint{0.000000in}{-0.048611in}}%
\pgfusepath{stroke,fill}%
}%
\begin{pgfscope}%
\pgfsys@transformshift{3.530505in}{0.528000in}%
\pgfsys@useobject{currentmarker}{}%
\end{pgfscope}%
\end{pgfscope}%
\begin{pgfscope}%
\definecolor{textcolor}{rgb}{0.000000,0.000000,0.000000}%
\pgfsetstrokecolor{textcolor}%
\pgfsetfillcolor{textcolor}%
\pgftext[x=3.530505in,y=0.430778in,,top]{\color{textcolor}\rmfamily\fontsize{10.000000}{12.000000}\selectfont \(\displaystyle {6}\)}%
\end{pgfscope}%
\begin{pgfscope}%
\pgfsetbuttcap%
\pgfsetroundjoin%
\definecolor{currentfill}{rgb}{0.000000,0.000000,0.000000}%
\pgfsetfillcolor{currentfill}%
\pgfsetlinewidth{0.803000pt}%
\definecolor{currentstroke}{rgb}{0.000000,0.000000,0.000000}%
\pgfsetstrokecolor{currentstroke}%
\pgfsetdash{}{0pt}%
\pgfsys@defobject{currentmarker}{\pgfqpoint{0.000000in}{-0.048611in}}{\pgfqpoint{0.000000in}{0.000000in}}{%
\pgfpathmoveto{\pgfqpoint{0.000000in}{0.000000in}}%
\pgfpathlineto{\pgfqpoint{0.000000in}{-0.048611in}}%
\pgfusepath{stroke,fill}%
}%
\begin{pgfscope}%
\pgfsys@transformshift{4.532525in}{0.528000in}%
\pgfsys@useobject{currentmarker}{}%
\end{pgfscope}%
\end{pgfscope}%
\begin{pgfscope}%
\definecolor{textcolor}{rgb}{0.000000,0.000000,0.000000}%
\pgfsetstrokecolor{textcolor}%
\pgfsetfillcolor{textcolor}%
\pgftext[x=4.532525in,y=0.430778in,,top]{\color{textcolor}\rmfamily\fontsize{10.000000}{12.000000}\selectfont \(\displaystyle {8}\)}%
\end{pgfscope}%
\begin{pgfscope}%
\pgfsetbuttcap%
\pgfsetroundjoin%
\definecolor{currentfill}{rgb}{0.000000,0.000000,0.000000}%
\pgfsetfillcolor{currentfill}%
\pgfsetlinewidth{0.803000pt}%
\definecolor{currentstroke}{rgb}{0.000000,0.000000,0.000000}%
\pgfsetstrokecolor{currentstroke}%
\pgfsetdash{}{0pt}%
\pgfsys@defobject{currentmarker}{\pgfqpoint{0.000000in}{-0.048611in}}{\pgfqpoint{0.000000in}{0.000000in}}{%
\pgfpathmoveto{\pgfqpoint{0.000000in}{0.000000in}}%
\pgfpathlineto{\pgfqpoint{0.000000in}{-0.048611in}}%
\pgfusepath{stroke,fill}%
}%
\begin{pgfscope}%
\pgfsys@transformshift{5.534545in}{0.528000in}%
\pgfsys@useobject{currentmarker}{}%
\end{pgfscope}%
\end{pgfscope}%
\begin{pgfscope}%
\definecolor{textcolor}{rgb}{0.000000,0.000000,0.000000}%
\pgfsetstrokecolor{textcolor}%
\pgfsetfillcolor{textcolor}%
\pgftext[x=5.534545in,y=0.430778in,,top]{\color{textcolor}\rmfamily\fontsize{10.000000}{12.000000}\selectfont \(\displaystyle {10}\)}%
\end{pgfscope}%
\begin{pgfscope}%
\pgfsetbuttcap%
\pgfsetroundjoin%
\definecolor{currentfill}{rgb}{0.000000,0.000000,0.000000}%
\pgfsetfillcolor{currentfill}%
\pgfsetlinewidth{0.803000pt}%
\definecolor{currentstroke}{rgb}{0.000000,0.000000,0.000000}%
\pgfsetstrokecolor{currentstroke}%
\pgfsetdash{}{0pt}%
\pgfsys@defobject{currentmarker}{\pgfqpoint{-0.048611in}{0.000000in}}{\pgfqpoint{-0.000000in}{0.000000in}}{%
\pgfpathmoveto{\pgfqpoint{-0.000000in}{0.000000in}}%
\pgfpathlineto{\pgfqpoint{-0.048611in}{0.000000in}}%
\pgfusepath{stroke,fill}%
}%
\begin{pgfscope}%
\pgfsys@transformshift{0.800000in}{0.810962in}%
\pgfsys@useobject{currentmarker}{}%
\end{pgfscope}%
\end{pgfscope}%
\begin{pgfscope}%
\definecolor{textcolor}{rgb}{0.000000,0.000000,0.000000}%
\pgfsetstrokecolor{textcolor}%
\pgfsetfillcolor{textcolor}%
\pgftext[x=0.633333in, y=0.762737in, left, base]{\color{textcolor}\rmfamily\fontsize{10.000000}{12.000000}\selectfont \(\displaystyle {0}\)}%
\end{pgfscope}%
\begin{pgfscope}%
\pgfsetbuttcap%
\pgfsetroundjoin%
\definecolor{currentfill}{rgb}{0.000000,0.000000,0.000000}%
\pgfsetfillcolor{currentfill}%
\pgfsetlinewidth{0.803000pt}%
\definecolor{currentstroke}{rgb}{0.000000,0.000000,0.000000}%
\pgfsetstrokecolor{currentstroke}%
\pgfsetdash{}{0pt}%
\pgfsys@defobject{currentmarker}{\pgfqpoint{-0.048611in}{0.000000in}}{\pgfqpoint{-0.000000in}{0.000000in}}{%
\pgfpathmoveto{\pgfqpoint{-0.000000in}{0.000000in}}%
\pgfpathlineto{\pgfqpoint{-0.048611in}{0.000000in}}%
\pgfusepath{stroke,fill}%
}%
\begin{pgfscope}%
\pgfsys@transformshift{0.800000in}{1.483173in}%
\pgfsys@useobject{currentmarker}{}%
\end{pgfscope}%
\end{pgfscope}%
\begin{pgfscope}%
\definecolor{textcolor}{rgb}{0.000000,0.000000,0.000000}%
\pgfsetstrokecolor{textcolor}%
\pgfsetfillcolor{textcolor}%
\pgftext[x=0.563888in, y=1.434948in, left, base]{\color{textcolor}\rmfamily\fontsize{10.000000}{12.000000}\selectfont \(\displaystyle {20}\)}%
\end{pgfscope}%
\begin{pgfscope}%
\pgfsetbuttcap%
\pgfsetroundjoin%
\definecolor{currentfill}{rgb}{0.000000,0.000000,0.000000}%
\pgfsetfillcolor{currentfill}%
\pgfsetlinewidth{0.803000pt}%
\definecolor{currentstroke}{rgb}{0.000000,0.000000,0.000000}%
\pgfsetstrokecolor{currentstroke}%
\pgfsetdash{}{0pt}%
\pgfsys@defobject{currentmarker}{\pgfqpoint{-0.048611in}{0.000000in}}{\pgfqpoint{-0.000000in}{0.000000in}}{%
\pgfpathmoveto{\pgfqpoint{-0.000000in}{0.000000in}}%
\pgfpathlineto{\pgfqpoint{-0.048611in}{0.000000in}}%
\pgfusepath{stroke,fill}%
}%
\begin{pgfscope}%
\pgfsys@transformshift{0.800000in}{2.155384in}%
\pgfsys@useobject{currentmarker}{}%
\end{pgfscope}%
\end{pgfscope}%
\begin{pgfscope}%
\definecolor{textcolor}{rgb}{0.000000,0.000000,0.000000}%
\pgfsetstrokecolor{textcolor}%
\pgfsetfillcolor{textcolor}%
\pgftext[x=0.563888in, y=2.107158in, left, base]{\color{textcolor}\rmfamily\fontsize{10.000000}{12.000000}\selectfont \(\displaystyle {40}\)}%
\end{pgfscope}%
\begin{pgfscope}%
\pgfsetbuttcap%
\pgfsetroundjoin%
\definecolor{currentfill}{rgb}{0.000000,0.000000,0.000000}%
\pgfsetfillcolor{currentfill}%
\pgfsetlinewidth{0.803000pt}%
\definecolor{currentstroke}{rgb}{0.000000,0.000000,0.000000}%
\pgfsetstrokecolor{currentstroke}%
\pgfsetdash{}{0pt}%
\pgfsys@defobject{currentmarker}{\pgfqpoint{-0.048611in}{0.000000in}}{\pgfqpoint{-0.000000in}{0.000000in}}{%
\pgfpathmoveto{\pgfqpoint{-0.000000in}{0.000000in}}%
\pgfpathlineto{\pgfqpoint{-0.048611in}{0.000000in}}%
\pgfusepath{stroke,fill}%
}%
\begin{pgfscope}%
\pgfsys@transformshift{0.800000in}{2.827594in}%
\pgfsys@useobject{currentmarker}{}%
\end{pgfscope}%
\end{pgfscope}%
\begin{pgfscope}%
\definecolor{textcolor}{rgb}{0.000000,0.000000,0.000000}%
\pgfsetstrokecolor{textcolor}%
\pgfsetfillcolor{textcolor}%
\pgftext[x=0.563888in, y=2.779369in, left, base]{\color{textcolor}\rmfamily\fontsize{10.000000}{12.000000}\selectfont \(\displaystyle {60}\)}%
\end{pgfscope}%
\begin{pgfscope}%
\pgfsetbuttcap%
\pgfsetroundjoin%
\definecolor{currentfill}{rgb}{0.000000,0.000000,0.000000}%
\pgfsetfillcolor{currentfill}%
\pgfsetlinewidth{0.803000pt}%
\definecolor{currentstroke}{rgb}{0.000000,0.000000,0.000000}%
\pgfsetstrokecolor{currentstroke}%
\pgfsetdash{}{0pt}%
\pgfsys@defobject{currentmarker}{\pgfqpoint{-0.048611in}{0.000000in}}{\pgfqpoint{-0.000000in}{0.000000in}}{%
\pgfpathmoveto{\pgfqpoint{-0.000000in}{0.000000in}}%
\pgfpathlineto{\pgfqpoint{-0.048611in}{0.000000in}}%
\pgfusepath{stroke,fill}%
}%
\begin{pgfscope}%
\pgfsys@transformshift{0.800000in}{3.499805in}%
\pgfsys@useobject{currentmarker}{}%
\end{pgfscope}%
\end{pgfscope}%
\begin{pgfscope}%
\definecolor{textcolor}{rgb}{0.000000,0.000000,0.000000}%
\pgfsetstrokecolor{textcolor}%
\pgfsetfillcolor{textcolor}%
\pgftext[x=0.563888in, y=3.451580in, left, base]{\color{textcolor}\rmfamily\fontsize{10.000000}{12.000000}\selectfont \(\displaystyle {80}\)}%
\end{pgfscope}%
\begin{pgfscope}%
\pgfsetbuttcap%
\pgfsetroundjoin%
\definecolor{currentfill}{rgb}{0.000000,0.000000,0.000000}%
\pgfsetfillcolor{currentfill}%
\pgfsetlinewidth{0.803000pt}%
\definecolor{currentstroke}{rgb}{0.000000,0.000000,0.000000}%
\pgfsetstrokecolor{currentstroke}%
\pgfsetdash{}{0pt}%
\pgfsys@defobject{currentmarker}{\pgfqpoint{-0.048611in}{0.000000in}}{\pgfqpoint{-0.000000in}{0.000000in}}{%
\pgfpathmoveto{\pgfqpoint{-0.000000in}{0.000000in}}%
\pgfpathlineto{\pgfqpoint{-0.048611in}{0.000000in}}%
\pgfusepath{stroke,fill}%
}%
\begin{pgfscope}%
\pgfsys@transformshift{0.800000in}{4.172016in}%
\pgfsys@useobject{currentmarker}{}%
\end{pgfscope}%
\end{pgfscope}%
\begin{pgfscope}%
\definecolor{textcolor}{rgb}{0.000000,0.000000,0.000000}%
\pgfsetstrokecolor{textcolor}%
\pgfsetfillcolor{textcolor}%
\pgftext[x=0.494444in, y=4.123791in, left, base]{\color{textcolor}\rmfamily\fontsize{10.000000}{12.000000}\selectfont \(\displaystyle {100}\)}%
\end{pgfscope}%
\begin{pgfscope}%
\pgfpathrectangle{\pgfqpoint{0.800000in}{0.528000in}}{\pgfqpoint{4.960000in}{3.696000in}}%
\pgfusepath{clip}%
\pgfsetrectcap%
\pgfsetroundjoin%
\pgfsetlinewidth{1.505625pt}%
\definecolor{currentstroke}{rgb}{0.000000,0.000000,1.000000}%
\pgfsetstrokecolor{currentstroke}%
\pgfsetdash{}{0pt}%
\pgfpathmoveto{\pgfqpoint{1.025455in}{0.696000in}}%
\pgfpathlineto{\pgfqpoint{1.085576in}{0.699710in}}%
\pgfpathlineto{\pgfqpoint{1.145697in}{0.705615in}}%
\pgfpathlineto{\pgfqpoint{1.205818in}{0.713471in}}%
\pgfpathlineto{\pgfqpoint{1.265939in}{0.723129in}}%
\pgfpathlineto{\pgfqpoint{1.326061in}{0.734495in}}%
\pgfpathlineto{\pgfqpoint{1.386182in}{0.747500in}}%
\pgfpathlineto{\pgfqpoint{1.446303in}{0.762092in}}%
\pgfpathlineto{\pgfqpoint{1.536485in}{0.786856in}}%
\pgfpathlineto{\pgfqpoint{1.626667in}{0.814945in}}%
\pgfpathlineto{\pgfqpoint{1.716848in}{0.846208in}}%
\pgfpathlineto{\pgfqpoint{1.807030in}{0.880485in}}%
\pgfpathlineto{\pgfqpoint{1.897212in}{0.917602in}}%
\pgfpathlineto{\pgfqpoint{1.987394in}{0.957370in}}%
\pgfpathlineto{\pgfqpoint{2.077576in}{0.999592in}}%
\pgfpathlineto{\pgfqpoint{2.167758in}{1.044068in}}%
\pgfpathlineto{\pgfqpoint{2.257939in}{1.090596in}}%
\pgfpathlineto{\pgfqpoint{2.378182in}{1.155490in}}%
\pgfpathlineto{\pgfqpoint{2.498424in}{1.223262in}}%
\pgfpathlineto{\pgfqpoint{2.618667in}{1.293545in}}%
\pgfpathlineto{\pgfqpoint{2.738909in}{1.366057in}}%
\pgfpathlineto{\pgfqpoint{2.889212in}{1.459554in}}%
\pgfpathlineto{\pgfqpoint{3.039515in}{1.556113in}}%
\pgfpathlineto{\pgfqpoint{3.189818in}{1.655874in}}%
\pgfpathlineto{\pgfqpoint{3.310061in}{1.738243in}}%
\pgfpathlineto{\pgfqpoint{3.430303in}{1.823215in}}%
\pgfpathlineto{\pgfqpoint{3.550545in}{1.911167in}}%
\pgfpathlineto{\pgfqpoint{3.670788in}{2.002532in}}%
\pgfpathlineto{\pgfqpoint{3.760970in}{2.073569in}}%
\pgfpathlineto{\pgfqpoint{3.851152in}{2.146976in}}%
\pgfpathlineto{\pgfqpoint{3.941333in}{2.222942in}}%
\pgfpathlineto{\pgfqpoint{4.031515in}{2.301643in}}%
\pgfpathlineto{\pgfqpoint{4.121697in}{2.383237in}}%
\pgfpathlineto{\pgfqpoint{4.211879in}{2.467858in}}%
\pgfpathlineto{\pgfqpoint{4.302061in}{2.555614in}}%
\pgfpathlineto{\pgfqpoint{4.392242in}{2.646582in}}%
\pgfpathlineto{\pgfqpoint{4.482424in}{2.740805in}}%
\pgfpathlineto{\pgfqpoint{4.572606in}{2.838292in}}%
\pgfpathlineto{\pgfqpoint{4.662788in}{2.939016in}}%
\pgfpathlineto{\pgfqpoint{4.752970in}{3.042916in}}%
\pgfpathlineto{\pgfqpoint{4.843152in}{3.149899in}}%
\pgfpathlineto{\pgfqpoint{4.933333in}{3.259843in}}%
\pgfpathlineto{\pgfqpoint{5.023515in}{3.372601in}}%
\pgfpathlineto{\pgfqpoint{5.143758in}{3.527030in}}%
\pgfpathlineto{\pgfqpoint{5.264000in}{3.685722in}}%
\pgfpathlineto{\pgfqpoint{5.384242in}{3.848220in}}%
\pgfpathlineto{\pgfqpoint{5.504485in}{4.014066in}}%
\pgfpathlineto{\pgfqpoint{5.534545in}{4.056000in}}%
\pgfpathlineto{\pgfqpoint{5.534545in}{4.056000in}}%
\pgfusepath{stroke}%
\end{pgfscope}%
\begin{pgfscope}%
\pgfsetrectcap%
\pgfsetmiterjoin%
\pgfsetlinewidth{0.803000pt}%
\definecolor{currentstroke}{rgb}{0.000000,0.000000,0.000000}%
\pgfsetstrokecolor{currentstroke}%
\pgfsetdash{}{0pt}%
\pgfpathmoveto{\pgfqpoint{0.800000in}{0.528000in}}%
\pgfpathlineto{\pgfqpoint{0.800000in}{4.224000in}}%
\pgfusepath{stroke}%
\end{pgfscope}%
\begin{pgfscope}%
\pgfsetrectcap%
\pgfsetmiterjoin%
\pgfsetlinewidth{0.803000pt}%
\definecolor{currentstroke}{rgb}{0.000000,0.000000,0.000000}%
\pgfsetstrokecolor{currentstroke}%
\pgfsetdash{}{0pt}%
\pgfpathmoveto{\pgfqpoint{5.760000in}{0.528000in}}%
\pgfpathlineto{\pgfqpoint{5.760000in}{4.224000in}}%
\pgfusepath{stroke}%
\end{pgfscope}%
\begin{pgfscope}%
\pgfsetrectcap%
\pgfsetmiterjoin%
\pgfsetlinewidth{0.803000pt}%
\definecolor{currentstroke}{rgb}{0.000000,0.000000,0.000000}%
\pgfsetstrokecolor{currentstroke}%
\pgfsetdash{}{0pt}%
\pgfpathmoveto{\pgfqpoint{0.800000in}{0.528000in}}%
\pgfpathlineto{\pgfqpoint{5.760000in}{0.528000in}}%
\pgfusepath{stroke}%
\end{pgfscope}%
\begin{pgfscope}%
\pgfsetrectcap%
\pgfsetmiterjoin%
\pgfsetlinewidth{0.803000pt}%
\definecolor{currentstroke}{rgb}{0.000000,0.000000,0.000000}%
\pgfsetstrokecolor{currentstroke}%
\pgfsetdash{}{0pt}%
\pgfpathmoveto{\pgfqpoint{0.800000in}{4.224000in}}%
\pgfpathlineto{\pgfqpoint{5.760000in}{4.224000in}}%
\pgfusepath{stroke}%
\end{pgfscope}%
\begin{pgfscope}%
\definecolor{textcolor}{rgb}{0.000000,0.000000,0.000000}%
\pgfsetstrokecolor{textcolor}%
\pgfsetfillcolor{textcolor}%
\pgftext[x=3.280000in,y=4.307333in,,base]{\color{textcolor}\rmfamily\fontsize{12.000000}{14.400000}\selectfont Darboux sums, N = 15}%
\end{pgfscope}%
\end{pgfpicture}%
\makeatother%
\endgroup%

	\caption{Sumy dolne i górne.}
\end{figure}

Wyznaczone, przybliżone pole przez:
\begin{itemize}
	\item sumy dolne: $\py{trunc(DARBOUX[0])}$
	\item sumy górne: $\py{trunc(DARBOUX[1])}$
\end{itemize} 

\subsection{Całkowanie numeryczne - przez trapezy}
Metoda całkowania numerycznego opiera się na przybliżaniu pola pod wykresem funkcji za pomocą trapezów. Wysokością trapezu jest odcinek $[x_{i-1}, x_i]$ a podstawami są odpowiednio wartości funkcji $f$ w punktach $x_{i-1}$ i $x_i$. Ta metoda w większości przypadków jest dużo dokładniejsza od sumowania dolnego i górnego, lecz nie idealna --- zakładamy bowiem, że funkcja na odpowiednich przedziałach $[x_{i-1}, x_i]$ jest liniowa. Jak w poprzedniej metodzie --- oczywiście --- jeśli odcinek podzielimy na więcej fragmentów, to otrzymamy dokładniejszy wynik.
\vspace{4cm}
\begin{figure}[ht]
	%% Creator: Matplotlib, PGF backend
%%
%% To include the figure in your LaTeX document, write
%%   \input{<filename>.pgf}
%%
%% Make sure the required packages are loaded in your preamble
%%   \usepackage{pgf}
%%
%% Also ensure that all the required font packages are loaded; for instance,
%% the lmodern package is sometimes necessary when using math font.
%%   \usepackage{lmodern}
%%
%% Figures using additional raster images can only be included by \input if
%% they are in the same directory as the main LaTeX file. For loading figures
%% from other directories you can use the `import` package
%%   \usepackage{import}
%%
%% and then include the figures with
%%   \import{<path to file>}{<filename>.pgf}
%%
%% Matplotlib used the following preamble
%%   
%%   \makeatletter\@ifpackageloaded{underscore}{}{\usepackage[strings]{underscore}}\makeatother
%%
\begingroup%
\makeatletter%
\begin{pgfpicture}%
\pgfpathrectangle{\pgfpointorigin}{\pgfqpoint{6.400000in}{4.800000in}}%
\pgfusepath{use as bounding box, clip}%
\begin{pgfscope}%
\pgfsetbuttcap%
\pgfsetmiterjoin%
\definecolor{currentfill}{rgb}{1.000000,1.000000,1.000000}%
\pgfsetfillcolor{currentfill}%
\pgfsetlinewidth{0.000000pt}%
\definecolor{currentstroke}{rgb}{1.000000,1.000000,1.000000}%
\pgfsetstrokecolor{currentstroke}%
\pgfsetdash{}{0pt}%
\pgfpathmoveto{\pgfqpoint{0.000000in}{0.000000in}}%
\pgfpathlineto{\pgfqpoint{6.400000in}{0.000000in}}%
\pgfpathlineto{\pgfqpoint{6.400000in}{4.800000in}}%
\pgfpathlineto{\pgfqpoint{0.000000in}{4.800000in}}%
\pgfpathlineto{\pgfqpoint{0.000000in}{0.000000in}}%
\pgfpathclose%
\pgfusepath{fill}%
\end{pgfscope}%
\begin{pgfscope}%
\pgfsetbuttcap%
\pgfsetmiterjoin%
\definecolor{currentfill}{rgb}{1.000000,1.000000,1.000000}%
\pgfsetfillcolor{currentfill}%
\pgfsetlinewidth{0.000000pt}%
\definecolor{currentstroke}{rgb}{0.000000,0.000000,0.000000}%
\pgfsetstrokecolor{currentstroke}%
\pgfsetstrokeopacity{0.000000}%
\pgfsetdash{}{0pt}%
\pgfpathmoveto{\pgfqpoint{0.800000in}{0.528000in}}%
\pgfpathlineto{\pgfqpoint{5.760000in}{0.528000in}}%
\pgfpathlineto{\pgfqpoint{5.760000in}{4.224000in}}%
\pgfpathlineto{\pgfqpoint{0.800000in}{4.224000in}}%
\pgfpathlineto{\pgfqpoint{0.800000in}{0.528000in}}%
\pgfpathclose%
\pgfusepath{fill}%
\end{pgfscope}%
\begin{pgfscope}%
\pgfpathrectangle{\pgfqpoint{0.800000in}{0.528000in}}{\pgfqpoint{4.960000in}{3.696000in}}%
\pgfusepath{clip}%
\pgfsetbuttcap%
\pgfsetmiterjoin%
\definecolor{currentfill}{rgb}{0.000000,0.000000,1.000000}%
\pgfsetfillcolor{currentfill}%
\pgfsetfillopacity{0.300000}%
\pgfsetlinewidth{1.003750pt}%
\definecolor{currentstroke}{rgb}{0.000000,0.000000,1.000000}%
\pgfsetstrokecolor{currentstroke}%
\pgfsetstrokeopacity{0.300000}%
\pgfsetdash{}{0pt}%
\pgfpathmoveto{\pgfqpoint{1.025455in}{0.810962in}}%
\pgfpathlineto{\pgfqpoint{1.025455in}{0.696000in}}%
\pgfpathlineto{\pgfqpoint{1.326061in}{0.734495in}}%
\pgfpathlineto{\pgfqpoint{1.326061in}{0.810962in}}%
\pgfpathlineto{\pgfqpoint{1.025455in}{0.810962in}}%
\pgfpathclose%
\pgfusepath{stroke,fill}%
\end{pgfscope}%
\begin{pgfscope}%
\pgfpathrectangle{\pgfqpoint{0.800000in}{0.528000in}}{\pgfqpoint{4.960000in}{3.696000in}}%
\pgfusepath{clip}%
\pgfsetbuttcap%
\pgfsetmiterjoin%
\definecolor{currentfill}{rgb}{0.000000,0.000000,1.000000}%
\pgfsetfillcolor{currentfill}%
\pgfsetfillopacity{0.300000}%
\pgfsetlinewidth{1.003750pt}%
\definecolor{currentstroke}{rgb}{0.000000,0.000000,1.000000}%
\pgfsetstrokecolor{currentstroke}%
\pgfsetstrokeopacity{0.300000}%
\pgfsetdash{}{0pt}%
\pgfpathmoveto{\pgfqpoint{1.326061in}{0.810962in}}%
\pgfpathlineto{\pgfqpoint{1.326061in}{0.734495in}}%
\pgfpathlineto{\pgfqpoint{1.626667in}{0.814945in}}%
\pgfpathlineto{\pgfqpoint{1.626667in}{0.810962in}}%
\pgfpathlineto{\pgfqpoint{1.326061in}{0.810962in}}%
\pgfpathclose%
\pgfusepath{stroke,fill}%
\end{pgfscope}%
\begin{pgfscope}%
\pgfpathrectangle{\pgfqpoint{0.800000in}{0.528000in}}{\pgfqpoint{4.960000in}{3.696000in}}%
\pgfusepath{clip}%
\pgfsetbuttcap%
\pgfsetmiterjoin%
\definecolor{currentfill}{rgb}{0.000000,0.000000,1.000000}%
\pgfsetfillcolor{currentfill}%
\pgfsetfillopacity{0.300000}%
\pgfsetlinewidth{1.003750pt}%
\definecolor{currentstroke}{rgb}{0.000000,0.000000,1.000000}%
\pgfsetstrokecolor{currentstroke}%
\pgfsetstrokeopacity{0.300000}%
\pgfsetdash{}{0pt}%
\pgfpathmoveto{\pgfqpoint{1.626667in}{0.810962in}}%
\pgfpathlineto{\pgfqpoint{1.626667in}{0.814945in}}%
\pgfpathlineto{\pgfqpoint{1.927273in}{0.930573in}}%
\pgfpathlineto{\pgfqpoint{1.927273in}{0.810962in}}%
\pgfpathlineto{\pgfqpoint{1.626667in}{0.810962in}}%
\pgfpathclose%
\pgfusepath{stroke,fill}%
\end{pgfscope}%
\begin{pgfscope}%
\pgfpathrectangle{\pgfqpoint{0.800000in}{0.528000in}}{\pgfqpoint{4.960000in}{3.696000in}}%
\pgfusepath{clip}%
\pgfsetbuttcap%
\pgfsetmiterjoin%
\definecolor{currentfill}{rgb}{0.000000,0.000000,1.000000}%
\pgfsetfillcolor{currentfill}%
\pgfsetfillopacity{0.300000}%
\pgfsetlinewidth{1.003750pt}%
\definecolor{currentstroke}{rgb}{0.000000,0.000000,1.000000}%
\pgfsetstrokecolor{currentstroke}%
\pgfsetstrokeopacity{0.300000}%
\pgfsetdash{}{0pt}%
\pgfpathmoveto{\pgfqpoint{1.927273in}{0.810962in}}%
\pgfpathlineto{\pgfqpoint{1.927273in}{0.930573in}}%
\pgfpathlineto{\pgfqpoint{2.227879in}{1.074871in}}%
\pgfpathlineto{\pgfqpoint{2.227879in}{0.810962in}}%
\pgfpathlineto{\pgfqpoint{1.927273in}{0.810962in}}%
\pgfpathclose%
\pgfusepath{stroke,fill}%
\end{pgfscope}%
\begin{pgfscope}%
\pgfpathrectangle{\pgfqpoint{0.800000in}{0.528000in}}{\pgfqpoint{4.960000in}{3.696000in}}%
\pgfusepath{clip}%
\pgfsetbuttcap%
\pgfsetmiterjoin%
\definecolor{currentfill}{rgb}{0.000000,0.000000,1.000000}%
\pgfsetfillcolor{currentfill}%
\pgfsetfillopacity{0.300000}%
\pgfsetlinewidth{1.003750pt}%
\definecolor{currentstroke}{rgb}{0.000000,0.000000,1.000000}%
\pgfsetstrokecolor{currentstroke}%
\pgfsetstrokeopacity{0.300000}%
\pgfsetdash{}{0pt}%
\pgfpathmoveto{\pgfqpoint{2.227879in}{0.810962in}}%
\pgfpathlineto{\pgfqpoint{2.227879in}{1.074871in}}%
\pgfpathlineto{\pgfqpoint{2.528485in}{1.240610in}}%
\pgfpathlineto{\pgfqpoint{2.528485in}{0.810962in}}%
\pgfpathlineto{\pgfqpoint{2.227879in}{0.810962in}}%
\pgfpathclose%
\pgfusepath{stroke,fill}%
\end{pgfscope}%
\begin{pgfscope}%
\pgfpathrectangle{\pgfqpoint{0.800000in}{0.528000in}}{\pgfqpoint{4.960000in}{3.696000in}}%
\pgfusepath{clip}%
\pgfsetbuttcap%
\pgfsetmiterjoin%
\definecolor{currentfill}{rgb}{0.000000,0.000000,1.000000}%
\pgfsetfillcolor{currentfill}%
\pgfsetfillopacity{0.300000}%
\pgfsetlinewidth{1.003750pt}%
\definecolor{currentstroke}{rgb}{0.000000,0.000000,1.000000}%
\pgfsetstrokecolor{currentstroke}%
\pgfsetstrokeopacity{0.300000}%
\pgfsetdash{}{0pt}%
\pgfpathmoveto{\pgfqpoint{2.528485in}{0.810962in}}%
\pgfpathlineto{\pgfqpoint{2.528485in}{1.240610in}}%
\pgfpathlineto{\pgfqpoint{2.829091in}{1.421786in}}%
\pgfpathlineto{\pgfqpoint{2.829091in}{0.810962in}}%
\pgfpathlineto{\pgfqpoint{2.528485in}{0.810962in}}%
\pgfpathclose%
\pgfusepath{stroke,fill}%
\end{pgfscope}%
\begin{pgfscope}%
\pgfpathrectangle{\pgfqpoint{0.800000in}{0.528000in}}{\pgfqpoint{4.960000in}{3.696000in}}%
\pgfusepath{clip}%
\pgfsetbuttcap%
\pgfsetmiterjoin%
\definecolor{currentfill}{rgb}{0.000000,0.000000,1.000000}%
\pgfsetfillcolor{currentfill}%
\pgfsetfillopacity{0.300000}%
\pgfsetlinewidth{1.003750pt}%
\definecolor{currentstroke}{rgb}{0.000000,0.000000,1.000000}%
\pgfsetstrokecolor{currentstroke}%
\pgfsetstrokeopacity{0.300000}%
\pgfsetdash{}{0pt}%
\pgfpathmoveto{\pgfqpoint{2.829091in}{0.810962in}}%
\pgfpathlineto{\pgfqpoint{2.829091in}{1.421786in}}%
\pgfpathlineto{\pgfqpoint{3.129697in}{1.615566in}}%
\pgfpathlineto{\pgfqpoint{3.129697in}{0.810962in}}%
\pgfpathlineto{\pgfqpoint{2.829091in}{0.810962in}}%
\pgfpathclose%
\pgfusepath{stroke,fill}%
\end{pgfscope}%
\begin{pgfscope}%
\pgfpathrectangle{\pgfqpoint{0.800000in}{0.528000in}}{\pgfqpoint{4.960000in}{3.696000in}}%
\pgfusepath{clip}%
\pgfsetbuttcap%
\pgfsetmiterjoin%
\definecolor{currentfill}{rgb}{0.000000,0.000000,1.000000}%
\pgfsetfillcolor{currentfill}%
\pgfsetfillopacity{0.300000}%
\pgfsetlinewidth{1.003750pt}%
\definecolor{currentstroke}{rgb}{0.000000,0.000000,1.000000}%
\pgfsetstrokecolor{currentstroke}%
\pgfsetstrokeopacity{0.300000}%
\pgfsetdash{}{0pt}%
\pgfpathmoveto{\pgfqpoint{3.129697in}{0.810962in}}%
\pgfpathlineto{\pgfqpoint{3.129697in}{1.615566in}}%
\pgfpathlineto{\pgfqpoint{3.430303in}{1.823215in}}%
\pgfpathlineto{\pgfqpoint{3.430303in}{0.810962in}}%
\pgfpathlineto{\pgfqpoint{3.129697in}{0.810962in}}%
\pgfpathclose%
\pgfusepath{stroke,fill}%
\end{pgfscope}%
\begin{pgfscope}%
\pgfpathrectangle{\pgfqpoint{0.800000in}{0.528000in}}{\pgfqpoint{4.960000in}{3.696000in}}%
\pgfusepath{clip}%
\pgfsetbuttcap%
\pgfsetmiterjoin%
\definecolor{currentfill}{rgb}{0.000000,0.000000,1.000000}%
\pgfsetfillcolor{currentfill}%
\pgfsetfillopacity{0.300000}%
\pgfsetlinewidth{1.003750pt}%
\definecolor{currentstroke}{rgb}{0.000000,0.000000,1.000000}%
\pgfsetstrokecolor{currentstroke}%
\pgfsetstrokeopacity{0.300000}%
\pgfsetdash{}{0pt}%
\pgfpathmoveto{\pgfqpoint{3.430303in}{0.810962in}}%
\pgfpathlineto{\pgfqpoint{3.430303in}{1.823215in}}%
\pgfpathlineto{\pgfqpoint{3.730909in}{2.049636in}}%
\pgfpathlineto{\pgfqpoint{3.730909in}{0.810962in}}%
\pgfpathlineto{\pgfqpoint{3.430303in}{0.810962in}}%
\pgfpathclose%
\pgfusepath{stroke,fill}%
\end{pgfscope}%
\begin{pgfscope}%
\pgfpathrectangle{\pgfqpoint{0.800000in}{0.528000in}}{\pgfqpoint{4.960000in}{3.696000in}}%
\pgfusepath{clip}%
\pgfsetbuttcap%
\pgfsetmiterjoin%
\definecolor{currentfill}{rgb}{0.000000,0.000000,1.000000}%
\pgfsetfillcolor{currentfill}%
\pgfsetfillopacity{0.300000}%
\pgfsetlinewidth{1.003750pt}%
\definecolor{currentstroke}{rgb}{0.000000,0.000000,1.000000}%
\pgfsetstrokecolor{currentstroke}%
\pgfsetstrokeopacity{0.300000}%
\pgfsetdash{}{0pt}%
\pgfpathmoveto{\pgfqpoint{3.730909in}{0.810962in}}%
\pgfpathlineto{\pgfqpoint{3.730909in}{2.049636in}}%
\pgfpathlineto{\pgfqpoint{4.031515in}{2.301643in}}%
\pgfpathlineto{\pgfqpoint{4.031515in}{0.810962in}}%
\pgfpathlineto{\pgfqpoint{3.730909in}{0.810962in}}%
\pgfpathclose%
\pgfusepath{stroke,fill}%
\end{pgfscope}%
\begin{pgfscope}%
\pgfpathrectangle{\pgfqpoint{0.800000in}{0.528000in}}{\pgfqpoint{4.960000in}{3.696000in}}%
\pgfusepath{clip}%
\pgfsetbuttcap%
\pgfsetmiterjoin%
\definecolor{currentfill}{rgb}{0.000000,0.000000,1.000000}%
\pgfsetfillcolor{currentfill}%
\pgfsetfillopacity{0.300000}%
\pgfsetlinewidth{1.003750pt}%
\definecolor{currentstroke}{rgb}{0.000000,0.000000,1.000000}%
\pgfsetstrokecolor{currentstroke}%
\pgfsetstrokeopacity{0.300000}%
\pgfsetdash{}{0pt}%
\pgfpathmoveto{\pgfqpoint{4.031515in}{0.810962in}}%
\pgfpathlineto{\pgfqpoint{4.031515in}{2.301643in}}%
\pgfpathlineto{\pgfqpoint{4.332121in}{2.585577in}}%
\pgfpathlineto{\pgfqpoint{4.332121in}{0.810962in}}%
\pgfpathlineto{\pgfqpoint{4.031515in}{0.810962in}}%
\pgfpathclose%
\pgfusepath{stroke,fill}%
\end{pgfscope}%
\begin{pgfscope}%
\pgfpathrectangle{\pgfqpoint{0.800000in}{0.528000in}}{\pgfqpoint{4.960000in}{3.696000in}}%
\pgfusepath{clip}%
\pgfsetbuttcap%
\pgfsetmiterjoin%
\definecolor{currentfill}{rgb}{0.000000,0.000000,1.000000}%
\pgfsetfillcolor{currentfill}%
\pgfsetfillopacity{0.300000}%
\pgfsetlinewidth{1.003750pt}%
\definecolor{currentstroke}{rgb}{0.000000,0.000000,1.000000}%
\pgfsetstrokecolor{currentstroke}%
\pgfsetstrokeopacity{0.300000}%
\pgfsetdash{}{0pt}%
\pgfpathmoveto{\pgfqpoint{4.332121in}{0.810962in}}%
\pgfpathlineto{\pgfqpoint{4.332121in}{2.585577in}}%
\pgfpathlineto{\pgfqpoint{4.632727in}{2.905084in}}%
\pgfpathlineto{\pgfqpoint{4.632727in}{0.810962in}}%
\pgfpathlineto{\pgfqpoint{4.332121in}{0.810962in}}%
\pgfpathclose%
\pgfusepath{stroke,fill}%
\end{pgfscope}%
\begin{pgfscope}%
\pgfpathrectangle{\pgfqpoint{0.800000in}{0.528000in}}{\pgfqpoint{4.960000in}{3.696000in}}%
\pgfusepath{clip}%
\pgfsetbuttcap%
\pgfsetmiterjoin%
\definecolor{currentfill}{rgb}{0.000000,0.000000,1.000000}%
\pgfsetfillcolor{currentfill}%
\pgfsetfillopacity{0.300000}%
\pgfsetlinewidth{1.003750pt}%
\definecolor{currentstroke}{rgb}{0.000000,0.000000,1.000000}%
\pgfsetstrokecolor{currentstroke}%
\pgfsetstrokeopacity{0.300000}%
\pgfsetdash{}{0pt}%
\pgfpathmoveto{\pgfqpoint{4.632727in}{0.810962in}}%
\pgfpathlineto{\pgfqpoint{4.632727in}{2.905084in}}%
\pgfpathlineto{\pgfqpoint{4.933333in}{3.259843in}}%
\pgfpathlineto{\pgfqpoint{4.933333in}{0.810962in}}%
\pgfpathlineto{\pgfqpoint{4.632727in}{0.810962in}}%
\pgfpathclose%
\pgfusepath{stroke,fill}%
\end{pgfscope}%
\begin{pgfscope}%
\pgfpathrectangle{\pgfqpoint{0.800000in}{0.528000in}}{\pgfqpoint{4.960000in}{3.696000in}}%
\pgfusepath{clip}%
\pgfsetbuttcap%
\pgfsetmiterjoin%
\definecolor{currentfill}{rgb}{0.000000,0.000000,1.000000}%
\pgfsetfillcolor{currentfill}%
\pgfsetfillopacity{0.300000}%
\pgfsetlinewidth{1.003750pt}%
\definecolor{currentstroke}{rgb}{0.000000,0.000000,1.000000}%
\pgfsetstrokecolor{currentstroke}%
\pgfsetstrokeopacity{0.300000}%
\pgfsetdash{}{0pt}%
\pgfpathmoveto{\pgfqpoint{4.933333in}{0.810962in}}%
\pgfpathlineto{\pgfqpoint{4.933333in}{3.259843in}}%
\pgfpathlineto{\pgfqpoint{5.233939in}{3.645674in}}%
\pgfpathlineto{\pgfqpoint{5.233939in}{0.810962in}}%
\pgfpathlineto{\pgfqpoint{4.933333in}{0.810962in}}%
\pgfpathclose%
\pgfusepath{stroke,fill}%
\end{pgfscope}%
\begin{pgfscope}%
\pgfpathrectangle{\pgfqpoint{0.800000in}{0.528000in}}{\pgfqpoint{4.960000in}{3.696000in}}%
\pgfusepath{clip}%
\pgfsetbuttcap%
\pgfsetmiterjoin%
\definecolor{currentfill}{rgb}{0.000000,0.000000,1.000000}%
\pgfsetfillcolor{currentfill}%
\pgfsetfillopacity{0.300000}%
\pgfsetlinewidth{1.003750pt}%
\definecolor{currentstroke}{rgb}{0.000000,0.000000,1.000000}%
\pgfsetstrokecolor{currentstroke}%
\pgfsetstrokeopacity{0.300000}%
\pgfsetdash{}{0pt}%
\pgfpathmoveto{\pgfqpoint{5.233939in}{0.810962in}}%
\pgfpathlineto{\pgfqpoint{5.233939in}{3.645674in}}%
\pgfpathlineto{\pgfqpoint{5.534545in}{4.056000in}}%
\pgfpathlineto{\pgfqpoint{5.534545in}{0.810962in}}%
\pgfpathlineto{\pgfqpoint{5.233939in}{0.810962in}}%
\pgfpathclose%
\pgfusepath{stroke,fill}%
\end{pgfscope}%
\begin{pgfscope}%
\pgfsetbuttcap%
\pgfsetroundjoin%
\definecolor{currentfill}{rgb}{0.000000,0.000000,0.000000}%
\pgfsetfillcolor{currentfill}%
\pgfsetlinewidth{0.803000pt}%
\definecolor{currentstroke}{rgb}{0.000000,0.000000,0.000000}%
\pgfsetstrokecolor{currentstroke}%
\pgfsetdash{}{0pt}%
\pgfsys@defobject{currentmarker}{\pgfqpoint{0.000000in}{-0.048611in}}{\pgfqpoint{0.000000in}{0.000000in}}{%
\pgfpathmoveto{\pgfqpoint{0.000000in}{0.000000in}}%
\pgfpathlineto{\pgfqpoint{0.000000in}{-0.048611in}}%
\pgfusepath{stroke,fill}%
}%
\begin{pgfscope}%
\pgfsys@transformshift{1.526465in}{0.528000in}%
\pgfsys@useobject{currentmarker}{}%
\end{pgfscope}%
\end{pgfscope}%
\begin{pgfscope}%
\definecolor{textcolor}{rgb}{0.000000,0.000000,0.000000}%
\pgfsetstrokecolor{textcolor}%
\pgfsetfillcolor{textcolor}%
\pgftext[x=1.526465in,y=0.430778in,,top]{\color{textcolor}\rmfamily\fontsize{10.000000}{12.000000}\selectfont \(\displaystyle {2}\)}%
\end{pgfscope}%
\begin{pgfscope}%
\pgfsetbuttcap%
\pgfsetroundjoin%
\definecolor{currentfill}{rgb}{0.000000,0.000000,0.000000}%
\pgfsetfillcolor{currentfill}%
\pgfsetlinewidth{0.803000pt}%
\definecolor{currentstroke}{rgb}{0.000000,0.000000,0.000000}%
\pgfsetstrokecolor{currentstroke}%
\pgfsetdash{}{0pt}%
\pgfsys@defobject{currentmarker}{\pgfqpoint{0.000000in}{-0.048611in}}{\pgfqpoint{0.000000in}{0.000000in}}{%
\pgfpathmoveto{\pgfqpoint{0.000000in}{0.000000in}}%
\pgfpathlineto{\pgfqpoint{0.000000in}{-0.048611in}}%
\pgfusepath{stroke,fill}%
}%
\begin{pgfscope}%
\pgfsys@transformshift{2.528485in}{0.528000in}%
\pgfsys@useobject{currentmarker}{}%
\end{pgfscope}%
\end{pgfscope}%
\begin{pgfscope}%
\definecolor{textcolor}{rgb}{0.000000,0.000000,0.000000}%
\pgfsetstrokecolor{textcolor}%
\pgfsetfillcolor{textcolor}%
\pgftext[x=2.528485in,y=0.430778in,,top]{\color{textcolor}\rmfamily\fontsize{10.000000}{12.000000}\selectfont \(\displaystyle {4}\)}%
\end{pgfscope}%
\begin{pgfscope}%
\pgfsetbuttcap%
\pgfsetroundjoin%
\definecolor{currentfill}{rgb}{0.000000,0.000000,0.000000}%
\pgfsetfillcolor{currentfill}%
\pgfsetlinewidth{0.803000pt}%
\definecolor{currentstroke}{rgb}{0.000000,0.000000,0.000000}%
\pgfsetstrokecolor{currentstroke}%
\pgfsetdash{}{0pt}%
\pgfsys@defobject{currentmarker}{\pgfqpoint{0.000000in}{-0.048611in}}{\pgfqpoint{0.000000in}{0.000000in}}{%
\pgfpathmoveto{\pgfqpoint{0.000000in}{0.000000in}}%
\pgfpathlineto{\pgfqpoint{0.000000in}{-0.048611in}}%
\pgfusepath{stroke,fill}%
}%
\begin{pgfscope}%
\pgfsys@transformshift{3.530505in}{0.528000in}%
\pgfsys@useobject{currentmarker}{}%
\end{pgfscope}%
\end{pgfscope}%
\begin{pgfscope}%
\definecolor{textcolor}{rgb}{0.000000,0.000000,0.000000}%
\pgfsetstrokecolor{textcolor}%
\pgfsetfillcolor{textcolor}%
\pgftext[x=3.530505in,y=0.430778in,,top]{\color{textcolor}\rmfamily\fontsize{10.000000}{12.000000}\selectfont \(\displaystyle {6}\)}%
\end{pgfscope}%
\begin{pgfscope}%
\pgfsetbuttcap%
\pgfsetroundjoin%
\definecolor{currentfill}{rgb}{0.000000,0.000000,0.000000}%
\pgfsetfillcolor{currentfill}%
\pgfsetlinewidth{0.803000pt}%
\definecolor{currentstroke}{rgb}{0.000000,0.000000,0.000000}%
\pgfsetstrokecolor{currentstroke}%
\pgfsetdash{}{0pt}%
\pgfsys@defobject{currentmarker}{\pgfqpoint{0.000000in}{-0.048611in}}{\pgfqpoint{0.000000in}{0.000000in}}{%
\pgfpathmoveto{\pgfqpoint{0.000000in}{0.000000in}}%
\pgfpathlineto{\pgfqpoint{0.000000in}{-0.048611in}}%
\pgfusepath{stroke,fill}%
}%
\begin{pgfscope}%
\pgfsys@transformshift{4.532525in}{0.528000in}%
\pgfsys@useobject{currentmarker}{}%
\end{pgfscope}%
\end{pgfscope}%
\begin{pgfscope}%
\definecolor{textcolor}{rgb}{0.000000,0.000000,0.000000}%
\pgfsetstrokecolor{textcolor}%
\pgfsetfillcolor{textcolor}%
\pgftext[x=4.532525in,y=0.430778in,,top]{\color{textcolor}\rmfamily\fontsize{10.000000}{12.000000}\selectfont \(\displaystyle {8}\)}%
\end{pgfscope}%
\begin{pgfscope}%
\pgfsetbuttcap%
\pgfsetroundjoin%
\definecolor{currentfill}{rgb}{0.000000,0.000000,0.000000}%
\pgfsetfillcolor{currentfill}%
\pgfsetlinewidth{0.803000pt}%
\definecolor{currentstroke}{rgb}{0.000000,0.000000,0.000000}%
\pgfsetstrokecolor{currentstroke}%
\pgfsetdash{}{0pt}%
\pgfsys@defobject{currentmarker}{\pgfqpoint{0.000000in}{-0.048611in}}{\pgfqpoint{0.000000in}{0.000000in}}{%
\pgfpathmoveto{\pgfqpoint{0.000000in}{0.000000in}}%
\pgfpathlineto{\pgfqpoint{0.000000in}{-0.048611in}}%
\pgfusepath{stroke,fill}%
}%
\begin{pgfscope}%
\pgfsys@transformshift{5.534545in}{0.528000in}%
\pgfsys@useobject{currentmarker}{}%
\end{pgfscope}%
\end{pgfscope}%
\begin{pgfscope}%
\definecolor{textcolor}{rgb}{0.000000,0.000000,0.000000}%
\pgfsetstrokecolor{textcolor}%
\pgfsetfillcolor{textcolor}%
\pgftext[x=5.534545in,y=0.430778in,,top]{\color{textcolor}\rmfamily\fontsize{10.000000}{12.000000}\selectfont \(\displaystyle {10}\)}%
\end{pgfscope}%
\begin{pgfscope}%
\pgfsetbuttcap%
\pgfsetroundjoin%
\definecolor{currentfill}{rgb}{0.000000,0.000000,0.000000}%
\pgfsetfillcolor{currentfill}%
\pgfsetlinewidth{0.803000pt}%
\definecolor{currentstroke}{rgb}{0.000000,0.000000,0.000000}%
\pgfsetstrokecolor{currentstroke}%
\pgfsetdash{}{0pt}%
\pgfsys@defobject{currentmarker}{\pgfqpoint{-0.048611in}{0.000000in}}{\pgfqpoint{-0.000000in}{0.000000in}}{%
\pgfpathmoveto{\pgfqpoint{-0.000000in}{0.000000in}}%
\pgfpathlineto{\pgfqpoint{-0.048611in}{0.000000in}}%
\pgfusepath{stroke,fill}%
}%
\begin{pgfscope}%
\pgfsys@transformshift{0.800000in}{0.810962in}%
\pgfsys@useobject{currentmarker}{}%
\end{pgfscope}%
\end{pgfscope}%
\begin{pgfscope}%
\definecolor{textcolor}{rgb}{0.000000,0.000000,0.000000}%
\pgfsetstrokecolor{textcolor}%
\pgfsetfillcolor{textcolor}%
\pgftext[x=0.633333in, y=0.762737in, left, base]{\color{textcolor}\rmfamily\fontsize{10.000000}{12.000000}\selectfont \(\displaystyle {0}\)}%
\end{pgfscope}%
\begin{pgfscope}%
\pgfsetbuttcap%
\pgfsetroundjoin%
\definecolor{currentfill}{rgb}{0.000000,0.000000,0.000000}%
\pgfsetfillcolor{currentfill}%
\pgfsetlinewidth{0.803000pt}%
\definecolor{currentstroke}{rgb}{0.000000,0.000000,0.000000}%
\pgfsetstrokecolor{currentstroke}%
\pgfsetdash{}{0pt}%
\pgfsys@defobject{currentmarker}{\pgfqpoint{-0.048611in}{0.000000in}}{\pgfqpoint{-0.000000in}{0.000000in}}{%
\pgfpathmoveto{\pgfqpoint{-0.000000in}{0.000000in}}%
\pgfpathlineto{\pgfqpoint{-0.048611in}{0.000000in}}%
\pgfusepath{stroke,fill}%
}%
\begin{pgfscope}%
\pgfsys@transformshift{0.800000in}{1.483173in}%
\pgfsys@useobject{currentmarker}{}%
\end{pgfscope}%
\end{pgfscope}%
\begin{pgfscope}%
\definecolor{textcolor}{rgb}{0.000000,0.000000,0.000000}%
\pgfsetstrokecolor{textcolor}%
\pgfsetfillcolor{textcolor}%
\pgftext[x=0.563888in, y=1.434948in, left, base]{\color{textcolor}\rmfamily\fontsize{10.000000}{12.000000}\selectfont \(\displaystyle {20}\)}%
\end{pgfscope}%
\begin{pgfscope}%
\pgfsetbuttcap%
\pgfsetroundjoin%
\definecolor{currentfill}{rgb}{0.000000,0.000000,0.000000}%
\pgfsetfillcolor{currentfill}%
\pgfsetlinewidth{0.803000pt}%
\definecolor{currentstroke}{rgb}{0.000000,0.000000,0.000000}%
\pgfsetstrokecolor{currentstroke}%
\pgfsetdash{}{0pt}%
\pgfsys@defobject{currentmarker}{\pgfqpoint{-0.048611in}{0.000000in}}{\pgfqpoint{-0.000000in}{0.000000in}}{%
\pgfpathmoveto{\pgfqpoint{-0.000000in}{0.000000in}}%
\pgfpathlineto{\pgfqpoint{-0.048611in}{0.000000in}}%
\pgfusepath{stroke,fill}%
}%
\begin{pgfscope}%
\pgfsys@transformshift{0.800000in}{2.155384in}%
\pgfsys@useobject{currentmarker}{}%
\end{pgfscope}%
\end{pgfscope}%
\begin{pgfscope}%
\definecolor{textcolor}{rgb}{0.000000,0.000000,0.000000}%
\pgfsetstrokecolor{textcolor}%
\pgfsetfillcolor{textcolor}%
\pgftext[x=0.563888in, y=2.107158in, left, base]{\color{textcolor}\rmfamily\fontsize{10.000000}{12.000000}\selectfont \(\displaystyle {40}\)}%
\end{pgfscope}%
\begin{pgfscope}%
\pgfsetbuttcap%
\pgfsetroundjoin%
\definecolor{currentfill}{rgb}{0.000000,0.000000,0.000000}%
\pgfsetfillcolor{currentfill}%
\pgfsetlinewidth{0.803000pt}%
\definecolor{currentstroke}{rgb}{0.000000,0.000000,0.000000}%
\pgfsetstrokecolor{currentstroke}%
\pgfsetdash{}{0pt}%
\pgfsys@defobject{currentmarker}{\pgfqpoint{-0.048611in}{0.000000in}}{\pgfqpoint{-0.000000in}{0.000000in}}{%
\pgfpathmoveto{\pgfqpoint{-0.000000in}{0.000000in}}%
\pgfpathlineto{\pgfqpoint{-0.048611in}{0.000000in}}%
\pgfusepath{stroke,fill}%
}%
\begin{pgfscope}%
\pgfsys@transformshift{0.800000in}{2.827594in}%
\pgfsys@useobject{currentmarker}{}%
\end{pgfscope}%
\end{pgfscope}%
\begin{pgfscope}%
\definecolor{textcolor}{rgb}{0.000000,0.000000,0.000000}%
\pgfsetstrokecolor{textcolor}%
\pgfsetfillcolor{textcolor}%
\pgftext[x=0.563888in, y=2.779369in, left, base]{\color{textcolor}\rmfamily\fontsize{10.000000}{12.000000}\selectfont \(\displaystyle {60}\)}%
\end{pgfscope}%
\begin{pgfscope}%
\pgfsetbuttcap%
\pgfsetroundjoin%
\definecolor{currentfill}{rgb}{0.000000,0.000000,0.000000}%
\pgfsetfillcolor{currentfill}%
\pgfsetlinewidth{0.803000pt}%
\definecolor{currentstroke}{rgb}{0.000000,0.000000,0.000000}%
\pgfsetstrokecolor{currentstroke}%
\pgfsetdash{}{0pt}%
\pgfsys@defobject{currentmarker}{\pgfqpoint{-0.048611in}{0.000000in}}{\pgfqpoint{-0.000000in}{0.000000in}}{%
\pgfpathmoveto{\pgfqpoint{-0.000000in}{0.000000in}}%
\pgfpathlineto{\pgfqpoint{-0.048611in}{0.000000in}}%
\pgfusepath{stroke,fill}%
}%
\begin{pgfscope}%
\pgfsys@transformshift{0.800000in}{3.499805in}%
\pgfsys@useobject{currentmarker}{}%
\end{pgfscope}%
\end{pgfscope}%
\begin{pgfscope}%
\definecolor{textcolor}{rgb}{0.000000,0.000000,0.000000}%
\pgfsetstrokecolor{textcolor}%
\pgfsetfillcolor{textcolor}%
\pgftext[x=0.563888in, y=3.451580in, left, base]{\color{textcolor}\rmfamily\fontsize{10.000000}{12.000000}\selectfont \(\displaystyle {80}\)}%
\end{pgfscope}%
\begin{pgfscope}%
\pgfsetbuttcap%
\pgfsetroundjoin%
\definecolor{currentfill}{rgb}{0.000000,0.000000,0.000000}%
\pgfsetfillcolor{currentfill}%
\pgfsetlinewidth{0.803000pt}%
\definecolor{currentstroke}{rgb}{0.000000,0.000000,0.000000}%
\pgfsetstrokecolor{currentstroke}%
\pgfsetdash{}{0pt}%
\pgfsys@defobject{currentmarker}{\pgfqpoint{-0.048611in}{0.000000in}}{\pgfqpoint{-0.000000in}{0.000000in}}{%
\pgfpathmoveto{\pgfqpoint{-0.000000in}{0.000000in}}%
\pgfpathlineto{\pgfqpoint{-0.048611in}{0.000000in}}%
\pgfusepath{stroke,fill}%
}%
\begin{pgfscope}%
\pgfsys@transformshift{0.800000in}{4.172016in}%
\pgfsys@useobject{currentmarker}{}%
\end{pgfscope}%
\end{pgfscope}%
\begin{pgfscope}%
\definecolor{textcolor}{rgb}{0.000000,0.000000,0.000000}%
\pgfsetstrokecolor{textcolor}%
\pgfsetfillcolor{textcolor}%
\pgftext[x=0.494444in, y=4.123791in, left, base]{\color{textcolor}\rmfamily\fontsize{10.000000}{12.000000}\selectfont \(\displaystyle {100}\)}%
\end{pgfscope}%
\begin{pgfscope}%
\pgfpathrectangle{\pgfqpoint{0.800000in}{0.528000in}}{\pgfqpoint{4.960000in}{3.696000in}}%
\pgfusepath{clip}%
\pgfsetrectcap%
\pgfsetroundjoin%
\pgfsetlinewidth{1.505625pt}%
\definecolor{currentstroke}{rgb}{0.121569,0.466667,0.705882}%
\pgfsetstrokecolor{currentstroke}%
\pgfsetdash{}{0pt}%
\pgfpathmoveto{\pgfqpoint{1.025455in}{0.696000in}}%
\pgfpathlineto{\pgfqpoint{1.082570in}{0.699470in}}%
\pgfpathlineto{\pgfqpoint{1.145697in}{0.705615in}}%
\pgfpathlineto{\pgfqpoint{1.211830in}{0.714357in}}%
\pgfpathlineto{\pgfqpoint{1.280970in}{0.725814in}}%
\pgfpathlineto{\pgfqpoint{1.350109in}{0.739503in}}%
\pgfpathlineto{\pgfqpoint{1.422255in}{0.756068in}}%
\pgfpathlineto{\pgfqpoint{1.494400in}{0.774877in}}%
\pgfpathlineto{\pgfqpoint{1.569552in}{0.796777in}}%
\pgfpathlineto{\pgfqpoint{1.644703in}{0.820948in}}%
\pgfpathlineto{\pgfqpoint{1.722861in}{0.848401in}}%
\pgfpathlineto{\pgfqpoint{1.804024in}{0.879296in}}%
\pgfpathlineto{\pgfqpoint{1.888194in}{0.913767in}}%
\pgfpathlineto{\pgfqpoint{1.975370in}{0.951921in}}%
\pgfpathlineto{\pgfqpoint{2.065552in}{0.993828in}}%
\pgfpathlineto{\pgfqpoint{2.161745in}{1.041037in}}%
\pgfpathlineto{\pgfqpoint{2.263952in}{1.093766in}}%
\pgfpathlineto{\pgfqpoint{2.372170in}{1.152174in}}%
\pgfpathlineto{\pgfqpoint{2.486400in}{1.216366in}}%
\pgfpathlineto{\pgfqpoint{2.609648in}{1.288194in}}%
\pgfpathlineto{\pgfqpoint{2.741915in}{1.367897in}}%
\pgfpathlineto{\pgfqpoint{2.880194in}{1.453858in}}%
\pgfpathlineto{\pgfqpoint{3.018473in}{1.542408in}}%
\pgfpathlineto{\pgfqpoint{3.153745in}{1.631622in}}%
\pgfpathlineto{\pgfqpoint{3.283006in}{1.719495in}}%
\pgfpathlineto{\pgfqpoint{3.403248in}{1.803851in}}%
\pgfpathlineto{\pgfqpoint{3.517479in}{1.886659in}}%
\pgfpathlineto{\pgfqpoint{3.625697in}{1.967841in}}%
\pgfpathlineto{\pgfqpoint{3.727903in}{2.047257in}}%
\pgfpathlineto{\pgfqpoint{3.827103in}{2.127158in}}%
\pgfpathlineto{\pgfqpoint{3.923297in}{2.207536in}}%
\pgfpathlineto{\pgfqpoint{4.016485in}{2.288330in}}%
\pgfpathlineto{\pgfqpoint{4.106667in}{2.369431in}}%
\pgfpathlineto{\pgfqpoint{4.196848in}{2.453539in}}%
\pgfpathlineto{\pgfqpoint{4.287030in}{2.540767in}}%
\pgfpathlineto{\pgfqpoint{4.377212in}{2.631196in}}%
\pgfpathlineto{\pgfqpoint{4.467394in}{2.724875in}}%
\pgfpathlineto{\pgfqpoint{4.557576in}{2.821818in}}%
\pgfpathlineto{\pgfqpoint{4.647758in}{2.922006in}}%
\pgfpathlineto{\pgfqpoint{4.740945in}{3.028882in}}%
\pgfpathlineto{\pgfqpoint{4.837139in}{3.142673in}}%
\pgfpathlineto{\pgfqpoint{4.936339in}{3.263557in}}%
\pgfpathlineto{\pgfqpoint{5.038545in}{3.391656in}}%
\pgfpathlineto{\pgfqpoint{5.146764in}{3.530947in}}%
\pgfpathlineto{\pgfqpoint{5.260994in}{3.681707in}}%
\pgfpathlineto{\pgfqpoint{5.381236in}{3.844115in}}%
\pgfpathlineto{\pgfqpoint{5.510497in}{4.022438in}}%
\pgfpathlineto{\pgfqpoint{5.534545in}{4.056000in}}%
\pgfpathlineto{\pgfqpoint{5.534545in}{4.056000in}}%
\pgfusepath{stroke}%
\end{pgfscope}%
\begin{pgfscope}%
\pgfsetrectcap%
\pgfsetmiterjoin%
\pgfsetlinewidth{0.803000pt}%
\definecolor{currentstroke}{rgb}{0.000000,0.000000,0.000000}%
\pgfsetstrokecolor{currentstroke}%
\pgfsetdash{}{0pt}%
\pgfpathmoveto{\pgfqpoint{0.800000in}{0.528000in}}%
\pgfpathlineto{\pgfqpoint{0.800000in}{4.224000in}}%
\pgfusepath{stroke}%
\end{pgfscope}%
\begin{pgfscope}%
\pgfsetrectcap%
\pgfsetmiterjoin%
\pgfsetlinewidth{0.803000pt}%
\definecolor{currentstroke}{rgb}{0.000000,0.000000,0.000000}%
\pgfsetstrokecolor{currentstroke}%
\pgfsetdash{}{0pt}%
\pgfpathmoveto{\pgfqpoint{5.760000in}{0.528000in}}%
\pgfpathlineto{\pgfqpoint{5.760000in}{4.224000in}}%
\pgfusepath{stroke}%
\end{pgfscope}%
\begin{pgfscope}%
\pgfsetrectcap%
\pgfsetmiterjoin%
\pgfsetlinewidth{0.803000pt}%
\definecolor{currentstroke}{rgb}{0.000000,0.000000,0.000000}%
\pgfsetstrokecolor{currentstroke}%
\pgfsetdash{}{0pt}%
\pgfpathmoveto{\pgfqpoint{0.800000in}{0.528000in}}%
\pgfpathlineto{\pgfqpoint{5.760000in}{0.528000in}}%
\pgfusepath{stroke}%
\end{pgfscope}%
\begin{pgfscope}%
\pgfsetrectcap%
\pgfsetmiterjoin%
\pgfsetlinewidth{0.803000pt}%
\definecolor{currentstroke}{rgb}{0.000000,0.000000,0.000000}%
\pgfsetstrokecolor{currentstroke}%
\pgfsetdash{}{0pt}%
\pgfpathmoveto{\pgfqpoint{0.800000in}{4.224000in}}%
\pgfpathlineto{\pgfqpoint{5.760000in}{4.224000in}}%
\pgfusepath{stroke}%
\end{pgfscope}%
\begin{pgfscope}%
\definecolor{textcolor}{rgb}{0.000000,0.000000,0.000000}%
\pgfsetstrokecolor{textcolor}%
\pgfsetfillcolor{textcolor}%
\pgftext[x=3.280000in,y=4.307333in,,base]{\color{textcolor}\rmfamily\fontsize{12.000000}{14.400000}\selectfont Trapezoid Rule, N = 15}%
\end{pgfscope}%
\end{pgfpicture}%
\makeatother%
\endgroup%

	\caption{Całka przybliżona trapezami}
\end{figure}

Wyznaczone, przybliżone pole przez sumę pól trapezów: $\py{trunc(TRAPEZOID)}$

\section{Porównanie wyników}
W tabeli przedstawione osiągnięte wyniki dla użytych metod obliczania całki oznaczonej, wyznaczone dla:

\vspace{1cm}
\begin{itemize}
	\item Wybrana funkcja: $f(x) = \py{sympy.latex(funkcja)}$
	\item Wybrany przedział: $(\py{a}, \py{b})$
	\item Odcinek podzielony na $\py{n}$ równych części
\end{itemize}
	
\vspace{1cm}
\begin{tabular}{ |p{3cm}||p{3cm}|p{3cm}|p{3cm}|p{3cm}| }
	\hline
	\multicolumn{5}{|c|}{Wyniki} \\
	\hline
	Metoda &Całka Riemanna &Sumy górne &Sumy dolne &Trapezy\\
	\hline
	Wynik   &$\py{trunc(RIEMANN)}$ &$\py{trunc(DARBOUX[1])}$ &$\py{trunc(DARBOUX[0])}$ &$\py{trunc(TRAPEZOID)}$ \\
	\hline
\end{tabular}
	
\end{document}